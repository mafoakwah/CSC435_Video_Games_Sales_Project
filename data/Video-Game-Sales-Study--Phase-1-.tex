% Options for packages loaded elsewhere
\PassOptionsToPackage{unicode}{hyperref}
\PassOptionsToPackage{hyphens}{url}
%
\documentclass[
]{article}
\usepackage{amsmath,amssymb}
\usepackage{lmodern}
\usepackage{iftex}
\ifPDFTeX
  \usepackage[T1]{fontenc}
  \usepackage[utf8]{inputenc}
  \usepackage{textcomp} % provide euro and other symbols
\else % if luatex or xetex
  \usepackage{unicode-math}
  \defaultfontfeatures{Scale=MatchLowercase}
  \defaultfontfeatures[\rmfamily]{Ligatures=TeX,Scale=1}
\fi
% Use upquote if available, for straight quotes in verbatim environments
\IfFileExists{upquote.sty}{\usepackage{upquote}}{}
\IfFileExists{microtype.sty}{% use microtype if available
  \usepackage[]{microtype}
  \UseMicrotypeSet[protrusion]{basicmath} % disable protrusion for tt fonts
}{}
\makeatletter
\@ifundefined{KOMAClassName}{% if non-KOMA class
  \IfFileExists{parskip.sty}{%
    \usepackage{parskip}
  }{% else
    \setlength{\parindent}{0pt}
    \setlength{\parskip}{6pt plus 2pt minus 1pt}}
}{% if KOMA class
  \KOMAoptions{parskip=half}}
\makeatother
\usepackage{xcolor}
\usepackage[margin=1in]{geometry}
\usepackage{color}
\usepackage{fancyvrb}
\newcommand{\VerbBar}{|}
\newcommand{\VERB}{\Verb[commandchars=\\\{\}]}
\DefineVerbatimEnvironment{Highlighting}{Verbatim}{commandchars=\\\{\}}
% Add ',fontsize=\small' for more characters per line
\usepackage{framed}
\definecolor{shadecolor}{RGB}{248,248,248}
\newenvironment{Shaded}{\begin{snugshade}}{\end{snugshade}}
\newcommand{\AlertTok}[1]{\textcolor[rgb]{0.94,0.16,0.16}{#1}}
\newcommand{\AnnotationTok}[1]{\textcolor[rgb]{0.56,0.35,0.01}{\textbf{\textit{#1}}}}
\newcommand{\AttributeTok}[1]{\textcolor[rgb]{0.77,0.63,0.00}{#1}}
\newcommand{\BaseNTok}[1]{\textcolor[rgb]{0.00,0.00,0.81}{#1}}
\newcommand{\BuiltInTok}[1]{#1}
\newcommand{\CharTok}[1]{\textcolor[rgb]{0.31,0.60,0.02}{#1}}
\newcommand{\CommentTok}[1]{\textcolor[rgb]{0.56,0.35,0.01}{\textit{#1}}}
\newcommand{\CommentVarTok}[1]{\textcolor[rgb]{0.56,0.35,0.01}{\textbf{\textit{#1}}}}
\newcommand{\ConstantTok}[1]{\textcolor[rgb]{0.00,0.00,0.00}{#1}}
\newcommand{\ControlFlowTok}[1]{\textcolor[rgb]{0.13,0.29,0.53}{\textbf{#1}}}
\newcommand{\DataTypeTok}[1]{\textcolor[rgb]{0.13,0.29,0.53}{#1}}
\newcommand{\DecValTok}[1]{\textcolor[rgb]{0.00,0.00,0.81}{#1}}
\newcommand{\DocumentationTok}[1]{\textcolor[rgb]{0.56,0.35,0.01}{\textbf{\textit{#1}}}}
\newcommand{\ErrorTok}[1]{\textcolor[rgb]{0.64,0.00,0.00}{\textbf{#1}}}
\newcommand{\ExtensionTok}[1]{#1}
\newcommand{\FloatTok}[1]{\textcolor[rgb]{0.00,0.00,0.81}{#1}}
\newcommand{\FunctionTok}[1]{\textcolor[rgb]{0.00,0.00,0.00}{#1}}
\newcommand{\ImportTok}[1]{#1}
\newcommand{\InformationTok}[1]{\textcolor[rgb]{0.56,0.35,0.01}{\textbf{\textit{#1}}}}
\newcommand{\KeywordTok}[1]{\textcolor[rgb]{0.13,0.29,0.53}{\textbf{#1}}}
\newcommand{\NormalTok}[1]{#1}
\newcommand{\OperatorTok}[1]{\textcolor[rgb]{0.81,0.36,0.00}{\textbf{#1}}}
\newcommand{\OtherTok}[1]{\textcolor[rgb]{0.56,0.35,0.01}{#1}}
\newcommand{\PreprocessorTok}[1]{\textcolor[rgb]{0.56,0.35,0.01}{\textit{#1}}}
\newcommand{\RegionMarkerTok}[1]{#1}
\newcommand{\SpecialCharTok}[1]{\textcolor[rgb]{0.00,0.00,0.00}{#1}}
\newcommand{\SpecialStringTok}[1]{\textcolor[rgb]{0.31,0.60,0.02}{#1}}
\newcommand{\StringTok}[1]{\textcolor[rgb]{0.31,0.60,0.02}{#1}}
\newcommand{\VariableTok}[1]{\textcolor[rgb]{0.00,0.00,0.00}{#1}}
\newcommand{\VerbatimStringTok}[1]{\textcolor[rgb]{0.31,0.60,0.02}{#1}}
\newcommand{\WarningTok}[1]{\textcolor[rgb]{0.56,0.35,0.01}{\textbf{\textit{#1}}}}
\usepackage{graphicx}
\makeatletter
\def\maxwidth{\ifdim\Gin@nat@width>\linewidth\linewidth\else\Gin@nat@width\fi}
\def\maxheight{\ifdim\Gin@nat@height>\textheight\textheight\else\Gin@nat@height\fi}
\makeatother
% Scale images if necessary, so that they will not overflow the page
% margins by default, and it is still possible to overwrite the defaults
% using explicit options in \includegraphics[width, height, ...]{}
\setkeys{Gin}{width=\maxwidth,height=\maxheight,keepaspectratio}
% Set default figure placement to htbp
\makeatletter
\def\fps@figure{htbp}
\makeatother
\setlength{\emergencystretch}{3em} % prevent overfull lines
\providecommand{\tightlist}{%
  \setlength{\itemsep}{0pt}\setlength{\parskip}{0pt}}
\setcounter{secnumdepth}{-\maxdimen} % remove section numbering
\ifLuaTeX
  \usepackage{selnolig}  % disable illegal ligatures
\fi
\IfFileExists{bookmark.sty}{\usepackage{bookmark}}{\usepackage{hyperref}}
\IfFileExists{xurl.sty}{\usepackage{xurl}}{} % add URL line breaks if available
\urlstyle{same} % disable monospaced font for URLs
\hypersetup{
  pdftitle={Video Game Sales Investment Study},
  pdfauthor={Matthew Afoakwah, Nathan Cummings, Jaye Evans, Nikola Ignjatovic, Kevin Mainello, Chris Weiner},
  hidelinks,
  pdfcreator={LaTeX via pandoc}}

\title{Video Game Sales Investment Study}
\author{Matthew Afoakwah, Nathan Cummings, Jaye Evans, Nikola
Ignjatovic, Kevin Mainello, Chris Weiner}
\date{March 28th, 2023}

\begin{document}
\maketitle

\hypertarget{scenerio}{%
\subsection{1. Scenerio}\label{scenerio}}

Suppose their is an investor who wants to invest in a gaming platform.
However before the investor invests, they want to know which gaming
platform between the years 2010 to 2016 has the best NA, EU, and Global
sales. The investor also wants to know what video game category produces
the most income between 2010 and 2016 while for each video game platform
so that they can better invest their money to better profit themselves
in the future. The investor also wants to see what developers create
more profitable games and the video game categories those developers
mainly develop. The investor wants an in-depth analyst on the video game
platforms, video game categories, and developers before they invest
their money.

\hypertarget{preparation}{%
\subsection{2.Preparation}\label{preparation}}

The data we are using was uploaded to Kaggle by SID\_TWR. SID\_TWR
stated that this dataset was scraped from VGChartz and Metacritic.

2.1 Load Tidyverse package Before we work on our data we first want to
load it into R, but first we must add two R packages into R before we
attempt to work with data.

\begin{Shaded}
\begin{Highlighting}[]
\CommentTok{\# install.packages(\textquotesingle{}Tidyverse\textquotesingle{})}
\FunctionTok{library}\NormalTok{(tidyverse)  }\CommentTok{\#Used to load tidyverse package into R.}
\end{Highlighting}
\end{Shaded}

\begin{verbatim}
## Warning: package 'tidyverse' was built under R version 4.2.3
\end{verbatim}

\begin{verbatim}
## Warning: package 'ggplot2' was built under R version 4.2.3
\end{verbatim}

\begin{verbatim}
## Warning: package 'tibble' was built under R version 4.2.3
\end{verbatim}

\begin{verbatim}
## Warning: package 'tidyr' was built under R version 4.2.3
\end{verbatim}

\begin{verbatim}
## Warning: package 'readr' was built under R version 4.2.3
\end{verbatim}

\begin{verbatim}
## Warning: package 'dplyr' was built under R version 4.2.3
\end{verbatim}

\begin{verbatim}
## Warning: package 'forcats' was built under R version 4.2.3
\end{verbatim}

\begin{verbatim}
## Warning: package 'lubridate' was built under R version 4.2.3
\end{verbatim}

\begin{verbatim}
## -- Attaching core tidyverse packages ------------------------ tidyverse 2.0.0 --
## v dplyr     1.1.1     v readr     2.1.4
## v forcats   1.0.0     v stringr   1.5.0
## v ggplot2   3.4.2     v tibble    3.2.1
## v lubridate 1.9.2     v tidyr     1.3.0
## v purrr     1.0.1     
## -- Conflicts ------------------------------------------ tidyverse_conflicts() --
## x dplyr::filter() masks stats::filter()
## x dplyr::lag()    masks stats::lag()
## i Use the ]8;;http://conflicted.r-lib.org/conflicted package]8;; to force all conflicts to become errors
\end{verbatim}

\begin{Shaded}
\begin{Highlighting}[]
\FunctionTok{library}\NormalTok{(readr)  }\CommentTok{\#Used to load readr package in R to use to read CSV file.}
\end{Highlighting}
\end{Shaded}

2.2 Import dataset\\
Now we want to use import our dataset into R using the read\_CSV
function that is from our readr package.

\begin{Shaded}
\begin{Highlighting}[]
\CommentTok{\# Reads the CSV file from the file location using read\_csv, and imports the}
\CommentTok{\# file into R, named Video\_Games\_Sales}
\NormalTok{Video\_Games\_Sales }\OtherTok{\textless{}{-}} \FunctionTok{read\_csv}\NormalTok{(}\StringTok{"data/Video\_Games\_Sales\_as\_at\_22\_Dec\_2016.csv"}\NormalTok{)}
\end{Highlighting}
\end{Shaded}

\begin{verbatim}
## Rows: 16719 Columns: 16
## -- Column specification --------------------------------------------------------
## Delimiter: ","
## chr (7): Name, Platform, Year_of_Release, Genre, Publisher, Developer, Rating
## dbl (9): NA_Sales, EU_Sales, JP_Sales, Other_Sales, Global_Sales, Critic_Sco...
## 
## i Use `spec()` to retrieve the full column specification for this data.
## i Specify the column types or set `show_col_types = FALSE` to quiet this message.
\end{verbatim}

2.3 Preview dataset Before we can work on our dataset, we must first see
if the dataset is relivant to our task. In order to accomplish this we
must explore the dataset in R.

\begin{Shaded}
\begin{Highlighting}[]
\CommentTok{\# we use the head() function to get a preview of the dataset.}
\FunctionTok{head}\NormalTok{(Video\_Games\_Sales)}
\end{Highlighting}
\end{Shaded}

\begin{verbatim}
## # A tibble: 6 x 16
##   Name     Platf~1 Year_~2 Genre Publi~3 NA_Sa~4 EU_Sa~5 JP_Sa~6 Other~7 Globa~8
##   <chr>    <chr>   <chr>   <chr> <chr>     <dbl>   <dbl>   <dbl>   <dbl>   <dbl>
## 1 Wii Spo~ Wii     2006    Spor~ Ninten~    41.4   29.0     3.77    8.45    82.5
## 2 Super M~ NES     1985    Plat~ Ninten~    29.1    3.58    6.81    0.77    40.2
## 3 Mario K~ Wii     2008    Raci~ Ninten~    15.7   12.8     3.79    3.29    35.5
## 4 Wii Spo~ Wii     2009    Spor~ Ninten~    15.6   10.9     3.28    2.95    32.8
## 5 Pokemon~ GB      1996    Role~ Ninten~    11.3    8.89   10.2     1       31.4
## 6 Tetris   GB      1989    Puzz~ Ninten~    23.2    2.26    4.22    0.58    30.3
## # ... with 6 more variables: Critic_Score <dbl>, Critic_Count <dbl>,
## #   User_Score <dbl>, User_Count <dbl>, Developer <chr>, Rating <chr>, and
## #   abbreviated variable names 1: Platform, 2: Year_of_Release, 3: Publisher,
## #   4: NA_Sales, 5: EU_Sales, 6: JP_Sales, 7: Other_Sales, 8: Global_Sales
\end{verbatim}

We have now seen some of the data set but before we can verify that the
data set is relevant we must explore the data set further. To accomplish
this we use the glimpse() function

\begin{Shaded}
\begin{Highlighting}[]
\CommentTok{\# shows a glimpse of the data set along with the attributes.}
\FunctionTok{glimpse}\NormalTok{(Video\_Games\_Sales)}
\end{Highlighting}
\end{Shaded}

\begin{verbatim}
## Rows: 16,719
## Columns: 16
## $ Name            <chr> "Wii Sports", "Super Mario Bros.", "Mario Kart Wii", "~
## $ Platform        <chr> "Wii", "NES", "Wii", "Wii", "GB", "GB", "DS", "Wii", "~
## $ Year_of_Release <chr> "2006", "1985", "2008", "2009", "1996", "1989", "2006"~
## $ Genre           <chr> "Sports", "Platform", "Racing", "Sports", "Role-Playin~
## $ Publisher       <chr> "Nintendo", "Nintendo", "Nintendo", "Nintendo", "Ninte~
## $ NA_Sales        <dbl> 41.36, 29.08, 15.68, 15.61, 11.27, 23.20, 11.28, 13.96~
## $ EU_Sales        <dbl> 28.96, 3.58, 12.76, 10.93, 8.89, 2.26, 9.14, 9.18, 6.9~
## $ JP_Sales        <dbl> 3.77, 6.81, 3.79, 3.28, 10.22, 4.22, 6.50, 2.93, 4.70,~
## $ Other_Sales     <dbl> 8.45, 0.77, 3.29, 2.95, 1.00, 0.58, 2.88, 2.84, 2.24, ~
## $ Global_Sales    <dbl> 82.53, 40.24, 35.52, 32.77, 31.37, 30.26, 29.80, 28.92~
## $ Critic_Score    <dbl> 76, NA, 82, 80, NA, NA, 89, 58, 87, NA, NA, 91, NA, 80~
## $ Critic_Count    <dbl> 51, NA, 73, 73, NA, NA, 65, 41, 80, NA, NA, 64, NA, 63~
## $ User_Score      <dbl> 8.0, NA, 8.3, 8.0, NA, NA, 8.5, 6.6, 8.4, NA, NA, 8.6,~
## $ User_Count      <dbl> 322, NA, 709, 192, NA, NA, 431, 129, 594, NA, NA, 464,~
## $ Developer       <chr> "Nintendo", NA, "Nintendo", "Nintendo", NA, NA, "Ninte~
## $ Rating          <chr> "E", NA, "E", "E", NA, NA, "E", "E", "E", NA, NA, "E",~
\end{verbatim}

Now we want to look at all the columns within the data set. We
accomplish this using the colnames() function.

\begin{Shaded}
\begin{Highlighting}[]
\CommentTok{\# Shows the name of the columns in the Video\_Games\_Sales data set.}
\FunctionTok{colnames}\NormalTok{(Video\_Games\_Sales)}
\end{Highlighting}
\end{Shaded}

\begin{verbatim}
##  [1] "Name"            "Platform"        "Year_of_Release" "Genre"          
##  [5] "Publisher"       "NA_Sales"        "EU_Sales"        "JP_Sales"       
##  [9] "Other_Sales"     "Global_Sales"    "Critic_Score"    "Critic_Count"   
## [13] "User_Score"      "User_Count"      "Developer"       "Rating"
\end{verbatim}

Now we want to determine the size of the data set so we use the dim()
function to determine the data sets size

\begin{Shaded}
\begin{Highlighting}[]
\CommentTok{\# After using the dim() function we see that the length of the data set is 16}
\CommentTok{\# and that there is 16719 rows.}
\FunctionTok{dim}\NormalTok{(Video\_Games\_Sales)}
\end{Highlighting}
\end{Shaded}

\begin{verbatim}
## [1] 16719    16
\end{verbatim}

To continue our exploration we wish to see the data type of each column,
in order to accomplish this we use the class function.

\begin{Shaded}
\begin{Highlighting}[]
\CommentTok{\# We use the sapply() function to apply the class function on each attribute of}
\CommentTok{\# our data set.}
\FunctionTok{sapply}\NormalTok{(Video\_Games\_Sales, class)}
\end{Highlighting}
\end{Shaded}

\begin{verbatim}
##            Name        Platform Year_of_Release           Genre       Publisher 
##     "character"     "character"     "character"     "character"     "character" 
##        NA_Sales        EU_Sales        JP_Sales     Other_Sales    Global_Sales 
##       "numeric"       "numeric"       "numeric"       "numeric"       "numeric" 
##    Critic_Score    Critic_Count      User_Score      User_Count       Developer 
##       "numeric"       "numeric"       "numeric"       "numeric"     "character" 
##          Rating 
##     "character"
\end{verbatim}

From viewing our data set we see that our data set is relevant to our
task and has given us insight to our data set which will allow us to
move on to the next step.

\hypertarget{data-cleaning}{%
\subsection{3. Data Cleaning}\label{data-cleaning}}

Though we can see that our data set has information that can be used for
our problem there is also irrelevant data in the data set, in order to
focus on the information that we need we will need to clean the data set
using transformations.

3.1 Since we are focusing on the video game sales between 2010 to 2016
we will shrink the data set by filtering it.

\begin{Shaded}
\begin{Highlighting}[]
\CommentTok{\# use the filter() function to remove all rows where the release year is not}
\CommentTok{\# between 2010 and 2016. Then we store it in a new data set called}
\CommentTok{\# gamesales2010\_2016}
\NormalTok{gamesales2010\_2016 }\OtherTok{\textless{}{-}} \FunctionTok{filter}\NormalTok{(Video\_Games\_Sales, Year\_of\_Release }\SpecialCharTok{\textgreater{}=} \DecValTok{2010}\NormalTok{)}
\NormalTok{gamesales2010\_2016 }\OtherTok{\textless{}{-}} \FunctionTok{filter}\NormalTok{(gamesales2010\_2016, Year\_of\_Release }\SpecialCharTok{\textless{}=} \DecValTok{2016}\NormalTok{)}
\CommentTok{\# We use the dim() function and see that the amount of rows in the data set.}
\FunctionTok{dim}\NormalTok{(gamesales2010\_2016)}
\end{Highlighting}
\end{Shaded}

\begin{verbatim}
## [1] 5277   16
\end{verbatim}

3.2 We will now filter all rows where the Year\_of\_Release is unknown.

\begin{Shaded}
\begin{Highlighting}[]
\CommentTok{\# The filter() function removes all values that are \textquotesingle{}N/A\textquotesingle{} in the}
\CommentTok{\# Year\_of\_Release column}
\NormalTok{gamesales2010\_2016 }\OtherTok{\textless{}{-}} \FunctionTok{filter}\NormalTok{(gamesales2010\_2016, Year\_of\_Release }\SpecialCharTok{!=} \StringTok{"N/A"}\NormalTok{)}
\CommentTok{\# nrow() prints out the number of rows in gamesales2010\_2016.}
\FunctionTok{nrow}\NormalTok{(gamesales2010\_2016)}
\end{Highlighting}
\end{Shaded}

\begin{verbatim}
## [1] 5277
\end{verbatim}

We have decided to remove all rows where the Year\_of\_Release is
``N/A'' because there are not too much ``N/A'' value's missing in the
Year\_of\_Release column which would greatly impact our data set,
because of this we are able to easily remove the ``N/A'' values from the
data set without having to worry about our results being greatly
influenced.

3.3 We want to find and remove any duplicated files so we use the
duplicated() function.

\begin{Shaded}
\begin{Highlighting}[]
\CommentTok{\# We use the duplicated() to find all rows where they are repeated occurences.}
\CommentTok{\# \#Then we use the ! to filter them out from the graph.}
\NormalTok{gamesales2010\_2016 }\OtherTok{\textless{}{-}}\NormalTok{ gamesales2010\_2016[}\SpecialCharTok{!}\FunctionTok{duplicated}\NormalTok{(gamesales2010\_2016), ]}
\CommentTok{\# We use \#nrow() function to see the amount of rows left in the data set.}
\FunctionTok{nrow}\NormalTok{(gamesales2010\_2016)}
\end{Highlighting}
\end{Shaded}

\begin{verbatim}
## [1] 5277
\end{verbatim}

After checking/removing all duplicated files in the data set we see view
how many rows are in the data set. We notice that the number of rows did
not decrease so there were no repeated values in the dataset.

3.4 We want to identify any outliers that are in our data set before we
can continue any further. To accomplish this what we want to do is use a
boxplot graph to identify any outliers in the graph. Because the numeric
columns we will be focusing on are Global and NA Sales we will only look
for outliers in these two columns.

\begin{Shaded}
\begin{Highlighting}[]
\CommentTok{\# We create a boxplot for the Global Sales in gamesales\_2010\_2016 then set the}
\CommentTok{\# title to \textquotesingle{}Global Sales BoxPlot\textquotesingle{}.}
\FunctionTok{boxplot}\NormalTok{(gamesales2010\_2016}\SpecialCharTok{$}\NormalTok{Global\_Sales, }\AttributeTok{main =} \StringTok{"Global Sales BoxPlot"}\NormalTok{)}
\end{Highlighting}
\end{Shaded}

\includegraphics{Video-Game-Sales-Study--Phase-1-_files/figure-latex/unnamed-chunk-12-1.pdf}

\begin{Shaded}
\begin{Highlighting}[]
\CommentTok{\# We create a new variable outliers, then use boxplot.stats() to assign our}
\CommentTok{\# outliers from Global sales to outliers.}
\NormalTok{outliers }\OtherTok{\textless{}{-}} \FunctionTok{boxplot.stats}\NormalTok{(gamesales2010\_2016}\SpecialCharTok{$}\NormalTok{Global\_Sales)}\SpecialCharTok{$}\NormalTok{out}
\end{Highlighting}
\end{Shaded}

This boxplot creates a visual representation of the max, min, and median
values in the Global sales of our data set.

We can see that there are some outliers in the Global Sales, this allows
us to easily remove them from our data set. However we will not remove
them as they will also not affect our progress moving forward. Now, we
will check the outliers for the NA Sales.

\begin{Shaded}
\begin{Highlighting}[]
\CommentTok{\# We create a boxplot for the NA Sales in gamesales\_2010\_2016 then set the}
\CommentTok{\# title to \textquotesingle{}NA Sales BoxPlot\textquotesingle{}.}
\FunctionTok{boxplot}\NormalTok{(gamesales2010\_2016}\SpecialCharTok{$}\NormalTok{NA\_Sales, }\AttributeTok{main =} \StringTok{"NA Sales BoxPlot"}\NormalTok{)}
\end{Highlighting}
\end{Shaded}

\includegraphics{Video-Game-Sales-Study--Phase-1-_files/figure-latex/unnamed-chunk-13-1.pdf}

\begin{Shaded}
\begin{Highlighting}[]
\CommentTok{\# We create a new variable outliersNA, then use boxplot.stats() to assign our}
\CommentTok{\# outliers from NA sales to outliersNA.}
\NormalTok{outliersNA }\OtherTok{\textless{}{-}} \FunctionTok{boxplot.stats}\NormalTok{(gamesales2010\_2016}\SpecialCharTok{$}\NormalTok{NA\_Sales)}\SpecialCharTok{$}\NormalTok{out}
\end{Highlighting}
\end{Shaded}

This boxplot creates a visual representation of the max, min, and median
values in the NA sales of our data set.

In this data set outliers are usually the top selling games or the ones
that sell very little. This is why we did not remove any outliers, as it
is valuable data and is relevant to what we will be searching for in the
data set.

\hypertarget{data-exploration}{%
\subsection{4. Data Exploration}\label{data-exploration}}

We now develop questions to explore the data set even further.

\begin{enumerate}
\def\labelenumi{\arabic{enumi}.}
\item
  Which video game platform made the most NA Sales from 2010 to 2016?
\item
  Which video game platform made the most Global Sales from 2010 to
  2016?
\item
  Which video game category had the highest NA sales?
\item
  Which video game category made the highest Global sales?
\item
  Which top five video game developers have the highest Global video
  game sales?
\item
  Which video game genre is the top five developers most profitable in?
\end{enumerate}

4.1 Before we can go and answer the questions we first want to explore
the unique attributes in the platform column to see if there are any
discrepancies that need to be fixed.

\begin{Shaded}
\begin{Highlighting}[]
\CommentTok{\# unique function lists the unique characters in the platform column.}
\FunctionTok{unique}\NormalTok{(gamesales2010\_2016}\SpecialCharTok{$}\NormalTok{Platform)}
\end{Highlighting}
\end{Shaded}

\begin{verbatim}
##  [1] "X360" "PS3"  "DS"   "PS4"  "3DS"  "Wii"  "XOne" "WiiU" "PC"   "PSP" 
## [11] "PSV"  "PS2"
\end{verbatim}

From looking at the unique attributes in the platform column we do have
irregularities so we need to recode the names in Platform using the
recode() function.

\begin{Shaded}
\begin{Highlighting}[]
\CommentTok{\# We use recode() to change the values of certain attributes to new attributes}
\CommentTok{\# such as Nintendo, PlayStation, and Xbox.}
\NormalTok{gamesales2010\_2016}\SpecialCharTok{$}\NormalTok{Platform }\OtherTok{\textless{}{-}} \FunctionTok{recode}\NormalTok{(gamesales2010\_2016}\SpecialCharTok{$}\NormalTok{Platform, }\AttributeTok{X360 =} \StringTok{"Xbox"}\NormalTok{,}
    \AttributeTok{PS3 =} \StringTok{"PlayStation"}\NormalTok{, }\AttributeTok{DS =} \StringTok{"Nintendo"}\NormalTok{, }\AttributeTok{PS4 =} \StringTok{"PlayStation"}\NormalTok{, }\StringTok{\textasciigrave{}}\AttributeTok{3DS}\StringTok{\textasciigrave{}} \OtherTok{=} \StringTok{"Nintendo"}\NormalTok{,}
    \AttributeTok{Wii =} \StringTok{"Nintendo"}\NormalTok{, }\AttributeTok{XOne =} \StringTok{"Xbox"}\NormalTok{, }\AttributeTok{WiiU =} \StringTok{"Nintendo"}\NormalTok{, }\AttributeTok{PSP =} \StringTok{"PlayStation"}\NormalTok{, }\AttributeTok{PSV =} \StringTok{"PlayStation"}\NormalTok{,}
    \AttributeTok{PS2 =} \StringTok{"PlayStation"}\NormalTok{)}
\CommentTok{\# We use unique()to list the unique attributes in the platform column.}
\FunctionTok{unique}\NormalTok{(gamesales2010\_2016}\SpecialCharTok{$}\NormalTok{Platform)}
\end{Highlighting}
\end{Shaded}

\begin{verbatim}
## [1] "Xbox"        "PlayStation" "Nintendo"    "PC"
\end{verbatim}

Now we have finished changing the attributes in the Platform column and
can now move on to our next task.

4.2 In order to answer the first question we must calculate the NA
revenue for each platform.

\begin{Shaded}
\begin{Highlighting}[]
\CommentTok{\# We move the gamesales2010\_2016 data into a new data set.}
\NormalTok{platform\_NA\_Sales }\OtherTok{\textless{}{-}}\NormalTok{ gamesales2010\_2016 }\SpecialCharTok{\%\textgreater{}\%}
    \CommentTok{\# We use the group\_by() to group all the platforms in the platform\_NA\_Sales}
    \CommentTok{\# together.}
\FunctionTok{group\_by}\NormalTok{(Platform) }\SpecialCharTok{\%\textgreater{}\%}
    \CommentTok{\# We use the summarize() to summarize the sum of NA\_sales for each}
    \CommentTok{\# different platform.}
\FunctionTok{summarize}\NormalTok{(}\AttributeTok{NA\_total =} \FunctionTok{sum}\NormalTok{(NA\_Sales)) }\SpecialCharTok{\%\textgreater{}\%}
    \CommentTok{\# We arrange the data set by decreasing order according to the NA\_total.}
\FunctionTok{arrange}\NormalTok{(}\FunctionTok{desc}\NormalTok{(NA\_total))}
\CommentTok{\# We use the head() function to preview the new data set.}
\FunctionTok{head}\NormalTok{(platform\_NA\_Sales)}
\end{Highlighting}
\end{Shaded}

\begin{verbatim}
## # A tibble: 4 x 2
##   Platform    NA_total
##   <chr>          <dbl>
## 1 Xbox           427. 
## 2 PlayStation    362. 
## 3 Nintendo       302. 
## 4 PC              39.1
\end{verbatim}

What this code does is that it sums up the total NA sales made between
2010 to 2016 for each video game platform. We create another table which
calculates the NA and Global sales for each platform for every different
year as well.

\begin{Shaded}
\begin{Highlighting}[]
\CommentTok{\# We move the gamesales2010\_2016 data into a new data set.}
\NormalTok{platform\_SalesbyYear }\OtherTok{\textless{}{-}}\NormalTok{ gamesales2010\_2016 }\SpecialCharTok{\%\textgreater{}\%}
    \CommentTok{\# We use the group\_by() to group all the Platform and Year\_of\_Release}
    \CommentTok{\# together.}
\FunctionTok{group\_by}\NormalTok{(Platform, Year\_of\_Release) }\SpecialCharTok{\%\textgreater{}\%}
    \CommentTok{\# We use the summarize() to summarize the sum of NA\_sales for each}
    \CommentTok{\# different platform, and year.}
\FunctionTok{summarize}\NormalTok{(}\AttributeTok{NA\_total =} \FunctionTok{sum}\NormalTok{(NA\_Sales), }\AttributeTok{glbl\_total =} \FunctionTok{sum}\NormalTok{(Global\_Sales)) }\SpecialCharTok{\%\textgreater{}\%}
    \CommentTok{\# We arrange the data set by decreasing order according to the glbl\_total.}
\FunctionTok{arrange}\NormalTok{(}\FunctionTok{desc}\NormalTok{(glbl\_total))}
\end{Highlighting}
\end{Shaded}

\begin{verbatim}
## `summarise()` has grouped output by 'Platform'. You can override using the
## `.groups` argument.
\end{verbatim}

\begin{Shaded}
\begin{Highlighting}[]
\CommentTok{\# We use the head() function to preview the new data set.}
\FunctionTok{head}\NormalTok{(platform\_SalesbyYear)}
\end{Highlighting}
\end{Shaded}

\begin{verbatim}
## # A tibble: 6 x 4
## # Groups:   Platform [3]
##   Platform    Year_of_Release NA_total glbl_total
##   <chr>       <chr>              <dbl>      <dbl>
## 1 Nintendo    2010               113.        213.
## 2 PlayStation 2010                70.6       183.
## 3 PlayStation 2011                68.6       180.
## 4 Xbox        2010               107.        170.
## 5 PlayStation 2014                53.1       160.
## 6 PlayStation 2013                54.1       153.
\end{verbatim}

What this code does is that it calculates the total NA and Global sales
of each platform based on the year. This will allow us to find a trend
in the profit of each platform.

4.3 Secondly, we calculate the global revenue for each platform.

\begin{Shaded}
\begin{Highlighting}[]
\CommentTok{\# We move the gamesales2010\_2016 data into a new data set then pipe the}
\CommentTok{\# command.}
\NormalTok{platform\_glbl\_sales }\OtherTok{\textless{}{-}}\NormalTok{ gamesales2010\_2016 }\SpecialCharTok{\%\textgreater{}\%}
    \CommentTok{\# We use the group\_by() to group all the platforms together.}
\FunctionTok{group\_by}\NormalTok{(Platform) }\SpecialCharTok{\%\textgreater{}\%}
    \CommentTok{\# We use the summarize function to summarize the sum of Global\_sales for}
    \CommentTok{\# each different platform.}
\FunctionTok{summarize}\NormalTok{(}\AttributeTok{glbl\_total =} \FunctionTok{sum}\NormalTok{(Global\_Sales)) }\SpecialCharTok{\%\textgreater{}\%}
    \CommentTok{\# We arrange the data set by decreasing order according to the glbl\_total.}
\FunctionTok{arrange}\NormalTok{(}\FunctionTok{desc}\NormalTok{(glbl\_total))}
\CommentTok{\# We use the head() function to preview the table.}
\FunctionTok{head}\NormalTok{(platform\_glbl\_sales)}
\end{Highlighting}
\end{Shaded}

\begin{verbatim}
## # A tibble: 4 x 2
##   Platform    glbl_total
##   <chr>            <dbl>
## 1 PlayStation      1026.
## 2 Xbox              710.
## 3 Nintendo          687.
## 4 PC                122.
\end{verbatim}

4.4 Now, we want to calculate the NA revenue based on genre.but before
we can do that we must see if there are any dependencies with the
character values in genre.

\begin{Shaded}
\begin{Highlighting}[]
\CommentTok{\# We use the unique to search for any irregular values in the categories}
\CommentTok{\# column.}
\FunctionTok{unique}\NormalTok{(gamesales2010\_2016}\SpecialCharTok{$}\NormalTok{Genre)}
\end{Highlighting}
\end{Shaded}

\begin{verbatim}
##  [1] "Misc"         "Action"       "Role-Playing" "Shooter"      "Racing"      
##  [6] "Platform"     "Simulation"   "Sports"       "Fighting"     "Strategy"    
## [11] "Adventure"    "Puzzle"
\end{verbatim}

After using the unique() function we see that there are not really any
irregularities in the function which allows us to continue to calculate
the NA revenue based on genre.

\begin{Shaded}
\begin{Highlighting}[]
\CommentTok{\# We move the gamesales2010\_2016 data into a new data set.}
\NormalTok{genre\_na\_sales }\OtherTok{\textless{}{-}}\NormalTok{ gamesales2010\_2016 }\SpecialCharTok{\%\textgreater{}\%}
    \CommentTok{\# We use the group\_by() to group all the Genres together.}
\FunctionTok{group\_by}\NormalTok{(Genre) }\SpecialCharTok{\%\textgreater{}\%}
    \CommentTok{\# We use summarize() to summarize the sum of na\_sales for each different}
    \CommentTok{\# genre.}
\FunctionTok{summarise}\NormalTok{(}\AttributeTok{genre\_na\_total =} \FunctionTok{sum}\NormalTok{(NA\_Sales)) }\SpecialCharTok{\%\textgreater{}\%}
    \CommentTok{\# We arrange the data set by decreasing order according to the}
    \CommentTok{\# genre\_na\_total.}
\FunctionTok{arrange}\NormalTok{(}\FunctionTok{desc}\NormalTok{(genre\_na\_total))}
\CommentTok{\# We use the head() function to preview the table.}
\FunctionTok{head}\NormalTok{(genre\_na\_sales)}
\end{Highlighting}
\end{Shaded}

\begin{verbatim}
## # A tibble: 6 x 2
##   Genre        genre_na_total
##   <chr>                 <dbl>
## 1 Action                291. 
## 2 Shooter               237. 
## 3 Sports                157. 
## 4 Misc                  124. 
## 5 Role-Playing          112. 
## 6 Platform               54.9
\end{verbatim}

We now create another table that separates the NA sales by genre for
each year.

\begin{Shaded}
\begin{Highlighting}[]
\CommentTok{\# We move the gamesales2010\_2016 data into a new data set.}
\NormalTok{genre\_SalesbyYear }\OtherTok{\textless{}{-}}\NormalTok{ gamesales2010\_2016 }\SpecialCharTok{\%\textgreater{}\%}
    \CommentTok{\# We use the group\_by() to group all the Genre and Year\_of\_Release}
    \CommentTok{\# together.}
\FunctionTok{group\_by}\NormalTok{(Genre, Year\_of\_Release) }\SpecialCharTok{\%\textgreater{}\%}
    \CommentTok{\# We use the summarize() to summarize the sum of NA\_sales for each}
    \CommentTok{\# different Genre, and year.}
\FunctionTok{summarize}\NormalTok{(}\AttributeTok{NA\_total =} \FunctionTok{sum}\NormalTok{(NA\_Sales), }\AttributeTok{glbl\_total =} \FunctionTok{sum}\NormalTok{(Global\_Sales)) }\SpecialCharTok{\%\textgreater{}\%}
    \CommentTok{\# We arrange the data set by decreasing order according to the glbl\_total.}
\FunctionTok{arrange}\NormalTok{(}\FunctionTok{desc}\NormalTok{(glbl\_total))}
\end{Highlighting}
\end{Shaded}

\begin{verbatim}
## `summarise()` has grouped output by 'Genre'. You can override using the
## `.groups` argument.
\end{verbatim}

\begin{Shaded}
\begin{Highlighting}[]
\CommentTok{\# We use the head() function to preview the new data set.}
\FunctionTok{head}\NormalTok{(genre\_SalesbyYear)}
\end{Highlighting}
\end{Shaded}

\begin{verbatim}
## # A tibble: 6 x 4
## # Groups:   Genre [2]
##   Genre   Year_of_Release NA_total glbl_total
##   <chr>   <chr>              <dbl>      <dbl>
## 1 Action  2013                53.5      123. 
## 2 Action  2012                51.8      119. 
## 3 Action  2011                53.1      117. 
## 4 Action  2010                59.7      115. 
## 5 Shooter 2011                49.7       98.2
## 6 Action  2014                38.8       97.3
\end{verbatim}

4.5 We now want to calculate the Global revenue based on genre.

\begin{Shaded}
\begin{Highlighting}[]
\CommentTok{\# We move the gamesales2010\_2016 data into a new data set then pipe the}
\CommentTok{\# command.}
\NormalTok{genre\_glbl\_sales }\OtherTok{\textless{}{-}}\NormalTok{ gamesales2010\_2016 }\SpecialCharTok{\%\textgreater{}\%}
    \CommentTok{\# We use group\_by() to group all the Genres in the genre\_glbl\_sales}
    \CommentTok{\# together.}
\FunctionTok{group\_by}\NormalTok{(Genre) }\SpecialCharTok{\%\textgreater{}\%}
    \CommentTok{\# We use summarize() to summarize the sum of Global\_sales for each}
    \CommentTok{\# different genre.}
\FunctionTok{summarise}\NormalTok{(}\AttributeTok{genre\_glbl\_total =} \FunctionTok{sum}\NormalTok{(Global\_Sales)) }\SpecialCharTok{\%\textgreater{}\%}
    \CommentTok{\# We arrange the data set by decreasing order according to the}
    \CommentTok{\# genre\_na\_total.}
\FunctionTok{arrange}\NormalTok{(}\FunctionTok{desc}\NormalTok{(genre\_glbl\_total))}
\CommentTok{\# We use the head() function to preview the table.}
\FunctionTok{head}\NormalTok{(genre\_glbl\_sales)}
\end{Highlighting}
\end{Shaded}

\begin{verbatim}
## # A tibble: 6 x 2
##   Genre        genre_glbl_total
##   <chr>                   <dbl>
## 1 Action                   673.
## 2 Shooter                  480.
## 3 Sports                   329.
## 4 Role-Playing             315.
## 5 Misc                     235.
## 6 Racing                   123.
\end{verbatim}

4.6 Before we can attempt to calculate NA and Global sales based on
developer we must first determine whether there are any irregularities
in the Developer column.

\begin{Shaded}
\begin{Highlighting}[]
\CommentTok{\# We use unique() to take all unique values in the developer column then sort}
\CommentTok{\# them.}
\NormalTok{unique\_dev }\OtherTok{\textless{}{-}} \FunctionTok{sort}\NormalTok{(}\FunctionTok{unique}\NormalTok{(gamesales2010\_2016}\SpecialCharTok{$}\NormalTok{Developer))}
\end{Highlighting}
\end{Shaded}

After placing all the unique values from the Developer column of the
data set gamesales2010\_2016 in a new set, we see that there are over
800 different values and there's bound to be some missed spelled values
so we review all the unique values and change their names to match what
we want it to be. There are some values who may have be in a different
region but are part of a company but because we want the overall of a
company we change the different region companies to match its parent
company as well.

\begin{Shaded}
\begin{Highlighting}[]
\CommentTok{\# We use the recode() function from the dplyr package from our tidyverse}
\CommentTok{\# package in order change our variables.}
\NormalTok{gamesales2010\_2016}\SpecialCharTok{$}\NormalTok{Developer }\OtherTok{\textless{}{-}} \FunctionTok{recode}\NormalTok{(gamesales2010\_2016}\SpecialCharTok{$}\NormalTok{Developer, }\StringTok{\textasciigrave{}}\AttributeTok{1C: Maddox Games}\StringTok{\textasciigrave{}} \OtherTok{=} \StringTok{"1C Company"}\NormalTok{,}
    \StringTok{\textasciigrave{}}\AttributeTok{1C:Ino{-}Co}\StringTok{\textasciigrave{}} \OtherTok{=} \StringTok{"1C Company"}\NormalTok{, }\StringTok{\textasciigrave{}}\AttributeTok{2K Australia}\StringTok{\textasciigrave{}} \OtherTok{=} \StringTok{"2K Games"}\NormalTok{, }\StringTok{\textasciigrave{}}\AttributeTok{2K Czech}\StringTok{\textasciigrave{}} \OtherTok{=} \StringTok{"2K Games"}\NormalTok{,}
    \StringTok{\textasciigrave{}}\AttributeTok{2K Sports}\StringTok{\textasciigrave{}} \OtherTok{=} \StringTok{"2K Games"}\NormalTok{, }\StringTok{\textasciigrave{}}\AttributeTok{2K Marin}\StringTok{\textasciigrave{}} \OtherTok{=} \StringTok{"2K Games"}\NormalTok{, }\StringTok{\textasciigrave{}}\AttributeTok{2K Play}\StringTok{\textasciigrave{}} \OtherTok{=} \StringTok{"2K Games"}\NormalTok{, }\StringTok{\textasciigrave{}}\AttributeTok{505 Games, Sarbakan Inc.}\StringTok{\textasciigrave{}} \OtherTok{=} \StringTok{"505 Games"}\NormalTok{,}
    \StringTok{\textasciigrave{}}\AttributeTok{Activision, Behaviour Interactive}\StringTok{\textasciigrave{}} \OtherTok{=} \StringTok{"Activision"}\NormalTok{, }\StringTok{\textasciigrave{}}\AttributeTok{Activision, FreeStyleGames}\StringTok{\textasciigrave{}} \OtherTok{=} \StringTok{"Activision"}\NormalTok{,}
    \StringTok{\textasciigrave{}}\AttributeTok{Ambrella, The Pokemon Company}\StringTok{\textasciigrave{}} \OtherTok{=} \StringTok{"Ambrella"}\NormalTok{, }\StringTok{\textasciigrave{}}\AttributeTok{Armature Studio, comcept}\StringTok{\textasciigrave{}} \OtherTok{=} \StringTok{"Armature Studio"}\NormalTok{,}
    \StringTok{\textasciigrave{}}\AttributeTok{Artificial Mind and Movement, EA Redwood Shores}\StringTok{\textasciigrave{}} \OtherTok{=} \StringTok{"Artificial Mind and Movement"}\NormalTok{,}
    \StringTok{\textasciigrave{}}\AttributeTok{Atari, Atari SA}\StringTok{\textasciigrave{}} \OtherTok{=} \StringTok{"Atari"}\NormalTok{, }\StringTok{\textasciigrave{}}\AttributeTok{Atari, Slightly Mad Studios, Atari SA}\StringTok{\textasciigrave{}} \OtherTok{=} \StringTok{"Atari"}\NormalTok{,}
    \StringTok{\textasciigrave{}}\AttributeTok{Atlus, Dingo Inc.}\StringTok{\textasciigrave{}} \OtherTok{=} \StringTok{"Atlus"}\NormalTok{, }\StringTok{\textasciigrave{}}\AttributeTok{Atomic Planet Entertainment}\StringTok{\textasciigrave{}} \OtherTok{=} \StringTok{"Atomic Games"}\NormalTok{,}
    \StringTok{\textasciigrave{}}\AttributeTok{Avalanche Software}\StringTok{\textasciigrave{}} \OtherTok{=} \StringTok{"Avalanche Studios"}\NormalTok{, }\StringTok{\textasciigrave{}}\AttributeTok{Bandai Namco Games, Artdink}\StringTok{\textasciigrave{}} \OtherTok{=} \StringTok{"Bandai Namco Games"}\NormalTok{,}
    \StringTok{\textasciigrave{}}\AttributeTok{Beenox, Other Ocean Interactive}\StringTok{\textasciigrave{}} \OtherTok{=} \StringTok{"Beenox"}\NormalTok{, }\StringTok{\textasciigrave{}}\AttributeTok{Big Blue Bubble Inc., Scholastic, Inc.}\StringTok{\textasciigrave{}} \OtherTok{=} \StringTok{"Big Blue Bubble Inc."}\NormalTok{,}
    \StringTok{\textasciigrave{}}\AttributeTok{Big Blue Bubble Inc., Scholastic, Inc.}\StringTok{\textasciigrave{}} \OtherTok{=} \StringTok{"Big Blue Bubble Inc."}\NormalTok{, }\StringTok{\textasciigrave{}}\AttributeTok{Blitz Games Studios}\StringTok{\textasciigrave{}} \OtherTok{=} \StringTok{"Blitz Games"}\NormalTok{,}
    \StringTok{\textasciigrave{}}\AttributeTok{Blue Byte, Related Designs}\StringTok{\textasciigrave{}} \OtherTok{=} \StringTok{"Blue Byte"}\NormalTok{, }\StringTok{\textasciigrave{}}\AttributeTok{Bungie Software, Bungie}\StringTok{\textasciigrave{}} \OtherTok{=} \StringTok{"Bungie"}\NormalTok{,}
    \StringTok{\textasciigrave{}}\AttributeTok{Capcom Vancouver}\StringTok{\textasciigrave{}} \OtherTok{=} \StringTok{"Capcom"}\NormalTok{, }\StringTok{\textasciigrave{}}\AttributeTok{Capcom, Pipeworks Software, Inc.}\StringTok{\textasciigrave{}} \OtherTok{=} \StringTok{"Capcom"}\NormalTok{)}

\NormalTok{gamesales2010\_2016}\SpecialCharTok{$}\NormalTok{Developer }\OtherTok{\textless{}{-}} \FunctionTok{recode}\NormalTok{(gamesales2010\_2016}\SpecialCharTok{$}\NormalTok{Developer, }\StringTok{\textasciigrave{}}\AttributeTok{Capcom, QLOC}\StringTok{\textasciigrave{}} \OtherTok{=} \StringTok{"Capcom"}\NormalTok{,}
    \StringTok{\textasciigrave{}}\AttributeTok{Climax Entertainment}\StringTok{\textasciigrave{}} \OtherTok{=} \StringTok{"Climax Studios"}\NormalTok{, }\StringTok{\textasciigrave{}}\AttributeTok{Climax Group}\StringTok{\textasciigrave{}} \OtherTok{=} \StringTok{"Climax Studios"}\NormalTok{,}
    \StringTok{\textasciigrave{}}\AttributeTok{Climax Group, Climax Studios}\StringTok{\textasciigrave{}} \OtherTok{=} \StringTok{"Climax Studios"}\NormalTok{, }\StringTok{\textasciigrave{}}\AttributeTok{Codemasters Birmingham}\StringTok{\textasciigrave{}} \OtherTok{=} \StringTok{"Codemasters"}\NormalTok{,}
    \StringTok{\textasciigrave{}}\AttributeTok{Compile Heart, GCREST}\StringTok{\textasciigrave{}} \OtherTok{=} \StringTok{"Compile Heart"}\NormalTok{, }\StringTok{\textasciigrave{}}\AttributeTok{Crave, DTP Entertainment}\StringTok{\textasciigrave{}} \OtherTok{=} \StringTok{"Crave"}\NormalTok{,}
    \StringTok{\textasciigrave{}}\AttributeTok{Crystal Dynamics, Nixxes Software}\StringTok{\textasciigrave{}} \OtherTok{=} \StringTok{"Crystal Dynamics"}\NormalTok{, }\StringTok{\textasciigrave{}}\AttributeTok{Cyanide, Cyanide Studios}\StringTok{\textasciigrave{}} \OtherTok{=} \StringTok{"Cyanide"}\NormalTok{,}
    \StringTok{\textasciigrave{}}\AttributeTok{CyberConnect2, Racjin}\StringTok{\textasciigrave{}} \OtherTok{=} \StringTok{"CyberConnect2"}\NormalTok{, }\StringTok{\textasciigrave{}}\AttributeTok{CyberPlanet Interactive Public Co., Ltd., Maximum Family Games}\StringTok{\textasciigrave{}} \OtherTok{=} \StringTok{"CyberPlanet Interactive Public Co., Ltd."}\NormalTok{,}
    \StringTok{\textasciigrave{}}\AttributeTok{Deep Silver Dambuster Studios}\StringTok{\textasciigrave{}} \OtherTok{=} \StringTok{"Deep Silver"}\NormalTok{, }\StringTok{\textasciigrave{}}\AttributeTok{Deep Silver, Keen Games}\StringTok{\textasciigrave{}} \OtherTok{=} \StringTok{"Deep Silver"}\NormalTok{,}
    \StringTok{\textasciigrave{}}\AttributeTok{Dimps Corporation, Dream Execution}\StringTok{\textasciigrave{}} \OtherTok{=} \StringTok{"Dimps Corporation"}\NormalTok{, }\StringTok{\textasciigrave{}}\AttributeTok{Dimps Corporation, SCE Japan Studio}\StringTok{\textasciigrave{}} \OtherTok{=} \StringTok{"Dimps Corporation"}\NormalTok{,}
    \StringTok{\textasciigrave{}}\AttributeTok{Disney Interactive Studios, Land Ho!}\StringTok{\textasciigrave{}} \OtherTok{=} \StringTok{"Disney Interactive Studios"}\NormalTok{, }\StringTok{\textasciigrave{}}\AttributeTok{EA Black Box}\StringTok{\textasciigrave{}} \OtherTok{=} \StringTok{"EA Games"}\NormalTok{,}
    \StringTok{\textasciigrave{}}\AttributeTok{EA Bright Light}\StringTok{\textasciigrave{}} \OtherTok{=} \StringTok{"EA Games"}\NormalTok{, }\StringTok{\textasciigrave{}}\AttributeTok{EA Canada}\StringTok{\textasciigrave{}} \OtherTok{=} \StringTok{"EA Games"}\NormalTok{, }\StringTok{\textasciigrave{}}\AttributeTok{EA Canada, EA Vancouver}\StringTok{\textasciigrave{}} \OtherTok{=} \StringTok{"EA Games"}\NormalTok{,}
    \StringTok{\textasciigrave{}}\AttributeTok{EA DICE}\StringTok{\textasciigrave{}} \OtherTok{=} \StringTok{"EA Games"}\NormalTok{, }\StringTok{\textasciigrave{}}\AttributeTok{EA DICE, Danger Close}\StringTok{\textasciigrave{}} \OtherTok{=} \StringTok{"EA Games"}\NormalTok{)}
\CommentTok{\#\textquotesingle{}EA Sports\textquotesingle{} = \textquotesingle{}EA Games\textquotesingle{}}

\NormalTok{gamesales2010\_2016}\SpecialCharTok{$}\NormalTok{Developer }\OtherTok{\textless{}{-}} \FunctionTok{recode}\NormalTok{(gamesales2010\_2016}\SpecialCharTok{$}\NormalTok{Developer, }\StringTok{\textasciigrave{}}\AttributeTok{EA Sports}\StringTok{\textasciigrave{}} \OtherTok{=} \StringTok{"EA Games"}\NormalTok{,}
    \StringTok{\textasciigrave{}}\AttributeTok{EA Montreal}\StringTok{\textasciigrave{}} \OtherTok{=} \StringTok{"EA Games"}\NormalTok{, }\StringTok{\textasciigrave{}}\AttributeTok{EA Redwood Shores}\StringTok{\textasciigrave{}} \OtherTok{=} \StringTok{"EA Games"}\NormalTok{, }\StringTok{\textasciigrave{}}\AttributeTok{EA Sports, EA Canada}\StringTok{\textasciigrave{}} \OtherTok{=} \StringTok{"EA Games"}\NormalTok{,}
    \StringTok{\textasciigrave{}}\AttributeTok{EA Sports, EA Vancouver}\StringTok{\textasciigrave{}} \OtherTok{=} \StringTok{"EA Games"}\NormalTok{, }\StringTok{\textasciigrave{}}\AttributeTok{EA Tiburon}\StringTok{\textasciigrave{}} \OtherTok{=} \StringTok{"EA Games"}\NormalTok{, }\StringTok{\textasciigrave{}}\AttributeTok{Eidos Montreal, Nixxes Software}\StringTok{\textasciigrave{}} \OtherTok{=} \StringTok{"Eidos Montreal"}\NormalTok{,}
    \StringTok{\textasciigrave{}}\AttributeTok{Engine Software, Re{-}Logic}\StringTok{\textasciigrave{}} \OtherTok{=} \StringTok{"Engine Software"}\NormalTok{, }\StringTok{\textasciigrave{}}\AttributeTok{Epic Games, People Can Fly}\StringTok{\textasciigrave{}} \OtherTok{=} \StringTok{"Epic Games"}\NormalTok{,}
    \StringTok{\textasciigrave{}}\AttributeTok{Farsight Studios, Crave}\StringTok{\textasciigrave{}} \OtherTok{=} \StringTok{"Farsight Studios"}\NormalTok{, }\StringTok{\textasciigrave{}}\AttributeTok{Gaijin Entertainment}\StringTok{\textasciigrave{}} \OtherTok{=} \StringTok{"Gaijin Games"}\NormalTok{,}
    \StringTok{\textasciigrave{}}\AttributeTok{Gearbox Software, 3D Realms}\StringTok{\textasciigrave{}} \OtherTok{=} \StringTok{"Gearbox Software"}\NormalTok{, }\StringTok{\textasciigrave{}}\AttributeTok{Gearbox Software, WayForward}\StringTok{\textasciigrave{}} \OtherTok{=} \StringTok{"Gearbox Software"}\NormalTok{,}
    \StringTok{\textasciigrave{}}\AttributeTok{Guerilla Cambridge}\StringTok{\textasciigrave{}} \OtherTok{=} \StringTok{"Guerrilla Cambridge"}\NormalTok{, }\AttributeTok{Guerilla =} \StringTok{"Guerilla Cambridge"}\NormalTok{)}


\NormalTok{gamesales2010\_2016}\SpecialCharTok{$}\NormalTok{Developer }\OtherTok{\textless{}{-}} \FunctionTok{recode}\NormalTok{(gamesales2010\_2016}\SpecialCharTok{$}\NormalTok{Developer, }\StringTok{\textasciigrave{}}\AttributeTok{Harmonix Music Systems, Demiurge}\StringTok{\textasciigrave{}} \OtherTok{=} \StringTok{"Harmonix Music Systems"}\NormalTok{,}
    \StringTok{\textasciigrave{}}\AttributeTok{Headup Games, Crenetic Studios}\StringTok{\textasciigrave{}} \OtherTok{=} \StringTok{"Headup Games"}\NormalTok{, }\StringTok{\textasciigrave{}}\AttributeTok{Konami Computer Entertainment Hawaii}\StringTok{\textasciigrave{}} \OtherTok{=} \StringTok{"Konami"}\NormalTok{,}
    \StringTok{\textasciigrave{}}\AttributeTok{Marvelous AQL}\StringTok{\textasciigrave{}} \OtherTok{=} \StringTok{"Marvelous Inc."}\NormalTok{, }\StringTok{\textasciigrave{}}\AttributeTok{Marvelous Entertainment}\StringTok{\textasciigrave{}} \OtherTok{=} \StringTok{"Marvelous Inc."}\NormalTok{,}
    \StringTok{\textasciigrave{}}\AttributeTok{Midway Studios {-} Austin}\StringTok{\textasciigrave{}} \OtherTok{=} \StringTok{"Midway"}\NormalTok{, }\StringTok{\textasciigrave{}}\AttributeTok{Monolith Soft}\StringTok{\textasciigrave{}} \OtherTok{=} \StringTok{"Monolith Productions"}\NormalTok{,}
    \StringTok{\textasciigrave{}}\AttributeTok{Monolith Soft, Banpresto}\StringTok{\textasciigrave{}} \OtherTok{=} \StringTok{"Monolith Productions"}\NormalTok{)}

\NormalTok{gamesales2010\_2016}\SpecialCharTok{$}\NormalTok{Developer }\OtherTok{\textless{}{-}} \FunctionTok{recode}\NormalTok{(gamesales2010\_2016}\SpecialCharTok{$}\NormalTok{Developer, }\StringTok{\textasciigrave{}}\AttributeTok{Namco Bandai Games America, Namco Bandai Games}\StringTok{\textasciigrave{}} \OtherTok{=} \StringTok{"Namco Bandai Games"}\NormalTok{,}
    \StringTok{\textasciigrave{}}\AttributeTok{Namco Bandai Games, Bandai Namco Games}\StringTok{\textasciigrave{}} \OtherTok{=} \StringTok{"Namco Bandai Games"}\NormalTok{, }\StringTok{\textasciigrave{}}\AttributeTok{Namco Bandai Games, Cellius}\StringTok{\textasciigrave{}} \OtherTok{=} \StringTok{"Namco Bandai Games"}\NormalTok{,}
    \StringTok{\textasciigrave{}}\AttributeTok{Namco Bandai Games, Monkey Bar Games}\StringTok{\textasciigrave{}} \OtherTok{=} \StringTok{"Namco Bandai Games"}\NormalTok{, }\StringTok{\textasciigrave{}}\AttributeTok{NATSUME ATARI Inc.}\StringTok{\textasciigrave{}} \OtherTok{=} \StringTok{"Natsume"}\NormalTok{,}
    \StringTok{\textasciigrave{}}\AttributeTok{Nintendo EAD Tokyo}\StringTok{\textasciigrave{}} \OtherTok{=} \StringTok{"Nintendo"}\NormalTok{, }\StringTok{\textasciigrave{}}\AttributeTok{Nintendo, Camelot Software Planning}\StringTok{\textasciigrave{}} \OtherTok{=} \StringTok{"Nintendo"}\NormalTok{,}
    \StringTok{\textasciigrave{}}\AttributeTok{Nintendo, Headstrong Games}\StringTok{\textasciigrave{}} \OtherTok{=} \StringTok{"Nintendo"}\NormalTok{, }\StringTok{\textasciigrave{}}\AttributeTok{Nintendo, Intelligent Systems}\StringTok{\textasciigrave{}} \OtherTok{=} \StringTok{"Nintendo"}\NormalTok{,}
    \StringTok{\textasciigrave{}}\AttributeTok{Nintendo, Nd Cube}\StringTok{\textasciigrave{}} \OtherTok{=} \StringTok{"Nintendo"}\NormalTok{, }\StringTok{\textasciigrave{}}\AttributeTok{Nintendo, Nintendo Software Technology}\StringTok{\textasciigrave{}} \OtherTok{=} \StringTok{"Nintendo"}\NormalTok{,}
    \StringTok{\textasciigrave{}}\AttributeTok{Nintendo, Spike Chunsoft}\StringTok{\textasciigrave{}} \OtherTok{=} \StringTok{"Nintendo"}\NormalTok{, }\StringTok{\textasciigrave{}}\AttributeTok{Paradox Development Studio}\StringTok{\textasciigrave{}} \OtherTok{=} \StringTok{"Paradox Interactive"}\NormalTok{)}


\NormalTok{gamesales2010\_2016}\SpecialCharTok{$}\NormalTok{Developer }\OtherTok{\textless{}{-}} \FunctionTok{recode}\NormalTok{(gamesales2010\_2016}\SpecialCharTok{$}\NormalTok{Developer, }\StringTok{\textasciigrave{}}\AttributeTok{PLAYGROUND, Playground Games}\StringTok{\textasciigrave{}} \OtherTok{=} \StringTok{"Playground Games"}\NormalTok{,}
    \StringTok{\textasciigrave{}}\AttributeTok{Retro Studios, Entertainment Analysis \& Development Division}\StringTok{\textasciigrave{}} \OtherTok{=} \StringTok{"Retro Studios"}\NormalTok{,}
    \StringTok{\textasciigrave{}}\AttributeTok{Rockstar Leeds}\StringTok{\textasciigrave{}} \OtherTok{=} \StringTok{"Rockstar Studios"}\NormalTok{, }\StringTok{\textasciigrave{}}\AttributeTok{Rockstar North}\StringTok{\textasciigrave{}} \OtherTok{=} \StringTok{"Rockstar Studios"}\NormalTok{,}
    \StringTok{\textasciigrave{}}\AttributeTok{Rockstar San Diego}\StringTok{\textasciigrave{}} \OtherTok{=} \StringTok{"Rockstar Studios"}\NormalTok{, }\StringTok{\textasciigrave{}}\AttributeTok{Sanzaru Games, Sanzaru Games, Inc.}\StringTok{\textasciigrave{}} \OtherTok{=} \StringTok{"Sanzaru Games"}\NormalTok{,}
    \StringTok{\textasciigrave{}}\AttributeTok{SCE Japan Studio, comcept}\StringTok{\textasciigrave{}} \OtherTok{=} \StringTok{"SCE Studio"}\NormalTok{, }\StringTok{\textasciigrave{}}\AttributeTok{SCE Santa Monica}\StringTok{\textasciigrave{}} \OtherTok{=} \StringTok{"SCE Studio"}\NormalTok{,}
    \StringTok{\textasciigrave{}}\AttributeTok{SCE Studio Cambridge}\StringTok{\textasciigrave{}} \OtherTok{=} \StringTok{"SCE Studio"}\NormalTok{, }\StringTok{\textasciigrave{}}\AttributeTok{SCE Japan Studio}\StringTok{\textasciigrave{}} \OtherTok{=} \StringTok{"SCE Studio"}\NormalTok{, }\StringTok{\textasciigrave{}}\AttributeTok{SCEA San Diego Studios}\StringTok{\textasciigrave{}} \OtherTok{=} \StringTok{"SCEA"}\NormalTok{,}
    \StringTok{\textasciigrave{}}\AttributeTok{SCEA, Zindagi Games}\StringTok{\textasciigrave{}} \OtherTok{=} \StringTok{"SCEA"}\NormalTok{, }\StringTok{\textasciigrave{}}\AttributeTok{SCEE London Studio}\StringTok{\textasciigrave{}} \OtherTok{=} \StringTok{"SCEE"}\NormalTok{)}

\NormalTok{gamesales2010\_2016}\SpecialCharTok{$}\NormalTok{Developer }\OtherTok{\textless{}{-}} \FunctionTok{recode}\NormalTok{(gamesales2010\_2016}\SpecialCharTok{$}\NormalTok{Developer, }\StringTok{\textasciigrave{}}\AttributeTok{Sega Studios San Francisco}\StringTok{\textasciigrave{}} \OtherTok{=} \StringTok{"Sega"}\NormalTok{,}
    \StringTok{\textasciigrave{}}\AttributeTok{Sega Toys}\StringTok{\textasciigrave{}} \OtherTok{=} \StringTok{"Sega"}\NormalTok{, }\StringTok{\textasciigrave{}}\AttributeTok{Sega, Dimps Corporation}\StringTok{\textasciigrave{}} \OtherTok{=} \StringTok{"Sega"}\NormalTok{, }\StringTok{\textasciigrave{}}\AttributeTok{Sega, French{-}Bread}\StringTok{\textasciigrave{}} \OtherTok{=} \StringTok{"Sega"}\NormalTok{,}
    \StringTok{\textasciigrave{}}\AttributeTok{Sega, Sonic Team}\StringTok{\textasciigrave{}} \OtherTok{=} \StringTok{"Sega"}\NormalTok{, }\AttributeTok{Snapdragon =} \StringTok{"Snap Dragon Games"}\NormalTok{, }\StringTok{\textasciigrave{}}\AttributeTok{Sonic Team}\StringTok{\textasciigrave{}} \OtherTok{=} \StringTok{"Sega"}\NormalTok{,}
    \StringTok{\textasciigrave{}}\AttributeTok{Sony Bend}\StringTok{\textasciigrave{}} \OtherTok{=} \StringTok{"Sony Interactive Entertainment"}\NormalTok{, }\StringTok{\textasciigrave{}}\AttributeTok{Sony Online Entertainment}\StringTok{\textasciigrave{}} \OtherTok{=} \StringTok{"Sony Interactive Entertainment"}\NormalTok{,}
    \StringTok{\textasciigrave{}}\AttributeTok{Spike Chunsoft}\StringTok{\textasciigrave{}} \OtherTok{=} \StringTok{"Spike"}\NormalTok{, }\StringTok{\textasciigrave{}}\AttributeTok{Spike Chunsoft Co. Ltd., Spike Chunsoft}\StringTok{\textasciigrave{}} \OtherTok{=} \StringTok{"Spike"}\NormalTok{,}
    \AttributeTok{Tecmo =} \StringTok{"Tecmo Koei Games"}\NormalTok{, }\StringTok{\textasciigrave{}}\AttributeTok{Tecmo Koei Canada}\StringTok{\textasciigrave{}} \OtherTok{=} \StringTok{"Tecmo Koei Games"}\NormalTok{, }\StringTok{\textasciigrave{}}\AttributeTok{THQ Australia}\StringTok{\textasciigrave{}} \OtherTok{=} \StringTok{"THQ"}\NormalTok{,}
    \StringTok{\textasciigrave{}}\AttributeTok{THQ Digital Studio Phoenix}\StringTok{\textasciigrave{}} \OtherTok{=} \StringTok{"THQ"}\NormalTok{, }\StringTok{\textasciigrave{}}\AttributeTok{Ubisoft Casablanca}\StringTok{\textasciigrave{}} \OtherTok{=} \StringTok{"Ubisoft"}\NormalTok{, }\StringTok{\textasciigrave{}}\AttributeTok{Ubisoft Milan}\StringTok{\textasciigrave{}} \OtherTok{=} \StringTok{"Ubisoft"}\NormalTok{,}
    \StringTok{\textasciigrave{}}\AttributeTok{Ubisoft Montpellier}\StringTok{\textasciigrave{}} \OtherTok{=} \StringTok{"Ubisoft"}\NormalTok{, }\StringTok{\textasciigrave{}}\AttributeTok{Ubisoft Montreal}\StringTok{\textasciigrave{}} \OtherTok{=} \StringTok{"Ubisoft"}\NormalTok{, }\StringTok{\textasciigrave{}}\AttributeTok{Ubisoft Osaka}\StringTok{\textasciigrave{}} \OtherTok{=} \StringTok{"Ubisoft"}\NormalTok{,}
    \StringTok{\textasciigrave{}}\AttributeTok{Ubisoft Paris}\StringTok{\textasciigrave{}} \OtherTok{=} \StringTok{"Ubisoft"}\NormalTok{, }\StringTok{\textasciigrave{}}\AttributeTok{Ubisoft Paris, Ubisoft Montpellier}\StringTok{\textasciigrave{}} \OtherTok{=} \StringTok{"Ubisoft"}\NormalTok{,}
    \StringTok{\textasciigrave{}}\AttributeTok{Ubisoft Quebec}\StringTok{\textasciigrave{}} \OtherTok{=} \StringTok{"Ubisoft"}\NormalTok{, }\StringTok{\textasciigrave{}}\AttributeTok{Ubisoft Reflections}\StringTok{\textasciigrave{}} \OtherTok{=} \StringTok{"Ubisoft"}\NormalTok{, }\StringTok{\textasciigrave{}}\AttributeTok{Ubisoft Reflections, Ivory Tower}\StringTok{\textasciigrave{}} \OtherTok{=} \StringTok{"Ubisoft"}\NormalTok{,}
    \StringTok{\textasciigrave{}}\AttributeTok{Ubisoft Romania}\StringTok{\textasciigrave{}} \OtherTok{=} \StringTok{"Ubisoft"}\NormalTok{, }\StringTok{\textasciigrave{}}\AttributeTok{Ubisoft Sofia}\StringTok{\textasciigrave{}} \OtherTok{=} \StringTok{"Ubisoft"}\NormalTok{, }\StringTok{\textasciigrave{}}\AttributeTok{Ubisoft Toronto}\StringTok{\textasciigrave{}} \OtherTok{=} \StringTok{"Ubisoft"}\NormalTok{,}
    \StringTok{\textasciigrave{}}\AttributeTok{Ubisoft Vancouver}\StringTok{\textasciigrave{}} \OtherTok{=} \StringTok{"Ubisoft"}\NormalTok{, }\StringTok{\textasciigrave{}}\AttributeTok{Ubisoft, FunHouse}\StringTok{\textasciigrave{}} \OtherTok{=} \StringTok{"Ubisoft"}\NormalTok{, }\StringTok{\textasciigrave{}}\AttributeTok{Ubisoft, Ludia Inc.}\StringTok{\textasciigrave{}} \OtherTok{=} \StringTok{"Ubisoft"}\NormalTok{,}
    \StringTok{\textasciigrave{}}\AttributeTok{Ubisoft, Ubisoft Montreal}\StringTok{\textasciigrave{}} \OtherTok{=} \StringTok{"Ubisoft"}\NormalTok{)}
\end{Highlighting}
\end{Shaded}

Because of how much values are in the Developer column there was many
values that needed to be changed in the dataset.

4.7 Since we have recoded the values in the Developers column we can now
calculate the global revenue based on the developer.

\begin{Shaded}
\begin{Highlighting}[]
\NormalTok{dev\_global }\OtherTok{\textless{}{-}}\NormalTok{ gamesales2010\_2016 }\SpecialCharTok{\%\textgreater{}\%}
    \CommentTok{\# We want to filter out any unknown Developers as there are so few NA}
    \CommentTok{\# developers that will affect our variables.}
\FunctionTok{filter}\NormalTok{(Developer }\SpecialCharTok{!=} \StringTok{"N/A"}\NormalTok{) }\SpecialCharTok{\%\textgreater{}\%}
    \CommentTok{\# We will then group the columns according to the value in the developer}
    \CommentTok{\# columns.}
\FunctionTok{group\_by}\NormalTok{(Developer) }\SpecialCharTok{\%\textgreater{}\%}
    \CommentTok{\# We sum up the total global sales and NA sales of each developers games}
    \CommentTok{\# earned.}
\FunctionTok{summarise}\NormalTok{(}\AttributeTok{global\_total =} \FunctionTok{sum}\NormalTok{(Global\_Sales)) }\SpecialCharTok{\%\textgreater{}\%}
    \CommentTok{\# We then arrange each row in descending order based on the global\_total.}
\FunctionTok{arrange}\NormalTok{(}\FunctionTok{desc}\NormalTok{(global\_total)) }\SpecialCharTok{\%\textgreater{}\%}
    \CommentTok{\# We use the slice\_head() function to take ONLY the top five most}
    \CommentTok{\# profitable developers.}
\FunctionTok{slice\_head}\NormalTok{(}\AttributeTok{n =} \DecValTok{5}\NormalTok{)}
\CommentTok{\# We preview the data set.}
\FunctionTok{head}\NormalTok{(dev\_global)}
\end{Highlighting}
\end{Shaded}

\begin{verbatim}
## # A tibble: 5 x 2
##   Developer        global_total
##   <chr>                   <dbl>
## 1 EA Games                209. 
## 2 Ubisoft                 183. 
## 3 Nintendo                 95.0
## 4 Rockstar Studios         75.7
## 5 Treyarch                 58.4
\end{verbatim}

4.8 Finally, we want to determine the video game categories that are
mainly developed by the top three developers and how the revenue in
these categories differ.

From the previous example we were able to already determine the top 5
developers, so if we use the information from the previous table we know
that ``EA Games'', ``Ubisoft'', ``Nintendo'', ``Rockstar Studios'', and
``Treyarch'' are the top five developers, this allows us to create a new
table which filters any developers that aren't those three developers.

\begin{Shaded}
\begin{Highlighting}[]
\CommentTok{\# We create a table which shows video games where the developers are the top}
\CommentTok{\# five developers.}
\NormalTok{developer\_genre\_sale }\OtherTok{\textless{}{-}}\NormalTok{ gamesales2010\_2016 }\SpecialCharTok{\%\textgreater{}\%}
    \FunctionTok{filter}\NormalTok{(Developer }\SpecialCharTok{\%in\%} \FunctionTok{c}\NormalTok{(}\StringTok{"EA Games"}\NormalTok{, }\StringTok{"Ubisoft"}\NormalTok{, }\StringTok{"Nintendo"}\NormalTok{, }\StringTok{"Rockstar Studios"}\NormalTok{,}
        \StringTok{"Treyarch"}\NormalTok{))}
\CommentTok{\# We use this to preview the data set.}
\FunctionTok{head}\NormalTok{(developer\_genre\_sale)}
\end{Highlighting}
\end{Shaded}

\begin{verbatim}
## # A tibble: 6 x 16
##   Name     Platf~1 Year_~2 Genre Publi~3 NA_Sa~4 EU_Sa~5 JP_Sa~6 Other~7 Globa~8
##   <chr>    <chr>   <chr>   <chr> <chr>     <dbl>   <dbl>   <dbl>   <dbl>   <dbl>
## 1 Grand T~ PlaySt~ 2013    Acti~ Take-T~    7.02    9.09    0.98    3.96    21.0
## 2 Grand T~ Xbox    2013    Acti~ Take-T~    9.66    5.14    0.06    1.41    16.3
## 3 Call of~ Xbox    2010    Shoo~ Activi~    9.7     3.68    0.11    1.13    14.6
## 4 Call of~ PlaySt~ 2012    Shoo~ Activi~    4.99    5.73    0.65    2.42    13.8
## 5 Call of~ Xbox    2012    Shoo~ Activi~    8.25    4.24    0.07    1.12    13.7
## 6 Call of~ PlaySt~ 2010    Shoo~ Activi~    5.99    4.37    0.48    1.79    12.6
## # ... with 6 more variables: Critic_Score <dbl>, Critic_Count <dbl>,
## #   User_Score <dbl>, User_Count <dbl>, Developer <chr>, Rating <chr>, and
## #   abbreviated variable names 1: Platform, 2: Year_of_Release, 3: Publisher,
## #   4: NA_Sales, 5: EU_Sales, 6: JP_Sales, 7: Other_Sales, 8: Global_Sales
\end{verbatim}

Now that we have isolated the video games that are developed by the top
five developers we can now group all the data by genre and developer and
calculate the global sale and NA sale based on the the the genre. but
first we must confirm the amount of N/A values in our Genre column to
see if those N/A's will affect our results.

\begin{Shaded}
\begin{Highlighting}[]
\CommentTok{\# We use is.na() to determine if any values in Genre is N/A. Then use sum() to}
\CommentTok{\# count the exact number of N/A values.}
\FunctionTok{sum}\NormalTok{(}\FunctionTok{is.na}\NormalTok{(developer\_genre\_sale}\SpecialCharTok{$}\NormalTok{Genre))}
\end{Highlighting}
\end{Shaded}

\begin{verbatim}
## [1] 0
\end{verbatim}

Fortunately, there is no N/A variables in the Genre column so we can
continue on with manipulating the data set.

\begin{Shaded}
\begin{Highlighting}[]
\CommentTok{\# We store the developer\_genre\_sale data into df.}
\NormalTok{df }\OtherTok{\textless{}{-}}\NormalTok{ developer\_genre\_sale }\SpecialCharTok{\%\textgreater{}\%}
    \CommentTok{\# We then group by Developer and Genre.}
\FunctionTok{group\_by}\NormalTok{(Developer, Genre) }\SpecialCharTok{\%\textgreater{}\%}
    \CommentTok{\# We summarise the NA and Global total for each category based on}
    \CommentTok{\# developer.}
\FunctionTok{summarise}\NormalTok{(}\AttributeTok{developer\_glbl\_total =} \FunctionTok{sum}\NormalTok{(Global\_Sales), }\AttributeTok{developer\_NA\_total =} \FunctionTok{sum}\NormalTok{(NA\_Sales)) }\SpecialCharTok{\%\textgreater{}\%}
    \CommentTok{\# We arrange the order in alphabetical order using the Developer column.}
\FunctionTok{arrange}\NormalTok{(Developer)}
\end{Highlighting}
\end{Shaded}

\begin{verbatim}
## `summarise()` has grouped output by 'Developer'. You can override using the
## `.groups` argument.
\end{verbatim}

\begin{Shaded}
\begin{Highlighting}[]
\CommentTok{\# We preview the data set.}
\FunctionTok{head}\NormalTok{(df)}
\end{Highlighting}
\end{Shaded}

\begin{verbatim}
## # A tibble: 6 x 4
## # Groups:   Developer [1]
##   Developer Genre     developer_glbl_total developer_NA_total
##   <chr>     <chr>                    <dbl>              <dbl>
## 1 EA Games  Action                    5.3                3.55
## 2 EA Games  Adventure                 0.32               0.17
## 3 EA Games  Fighting                  2.61               1.24
## 4 EA Games  Platform                  0.57               0.21
## 5 EA Games  Racing                    1.56               0.62
## 6 EA Games  Shooter                  49.7               23.1
\end{verbatim}

\hypertarget{analyzing}{%
\subsection{5. Analyzing}\label{analyzing}}

Now that we have cleaned the data set and explored the data set we can
now analyze the data set using visualization. In this section we will
use multiple different graphs which helps us identify and analyze our
data to help the client understand where they would receive the most
profit.

5.1 In order to be able to answer our first question, we must first
determine the most profitable platform and the rate of change between
sales of each platform since 2010 to 2016.

\begin{Shaded}
\begin{Highlighting}[]
\CommentTok{\# We use the ggplot2 package to create our graph, the aes() function is used to}
\CommentTok{\# set the aesthetics of the plot. we set our x value to Platform and our Y}
\CommentTok{\# value to NA\_total. we also use the reorder() function to reorder the graph}
\CommentTok{\# from smallest to greatest.}
\FunctionTok{ggplot}\NormalTok{(}\AttributeTok{data =}\NormalTok{ platform\_NA\_Sales, }\FunctionTok{aes}\NormalTok{(}\AttributeTok{x =} \FunctionTok{reorder}\NormalTok{(Platform, NA\_total), }\AttributeTok{y =}\NormalTok{ NA\_total)) }\SpecialCharTok{+}
    \FunctionTok{geom\_bar}\NormalTok{(}\AttributeTok{stat =} \StringTok{"identity"}\NormalTok{) }\SpecialCharTok{+} \FunctionTok{labs}\NormalTok{(}\AttributeTok{title =} \StringTok{"NA sales by Platform"}\NormalTok{, }\AttributeTok{x =} \StringTok{"Platform"}\NormalTok{,}
    \AttributeTok{y =} \StringTok{"NA Sales"}\NormalTok{) }\SpecialCharTok{+} \FunctionTok{theme}\NormalTok{(}\AttributeTok{axis.text.x =} \FunctionTok{element\_text}\NormalTok{(}\AttributeTok{angle =} \DecValTok{45}\NormalTok{, }\AttributeTok{hjust =} \DecValTok{1}\NormalTok{))}
\end{Highlighting}
\end{Shaded}

\includegraphics{Video-Game-Sales-Study--Phase-1-_files/figure-latex/unnamed-chunk-29-1.pdf}

\begin{Shaded}
\begin{Highlighting}[]
\CommentTok{\# labs() is used to changes the labels in various parts of the chart. Both the}
\CommentTok{\# X and Y axis changing their names, as well as the title. theme() helps change}
\CommentTok{\# non{-}data elements of a chart. In this case adding a change to Platform’s axis}
\CommentTok{\# by utilizing element\_text() to change the look of texts, adding an angle and}
\CommentTok{\# using hjust to control horizontal justification)}
\end{Highlighting}
\end{Shaded}

This code produces the above graph in which it displays the sum of all
Platforms in the 2010 to 2016 range.

\begin{Shaded}
\begin{Highlighting}[]
\CommentTok{\# We use the ggplot2 package to create our graph, the aes() function is used to}
\CommentTok{\# set the aesthetics of the plot. we set our x value to Year\_of\_Release, our Y}
\CommentTok{\# value to NA\_total, and fill based on platform.}
\FunctionTok{ggplot}\NormalTok{(platform\_SalesbyYear, }\FunctionTok{aes}\NormalTok{(}\AttributeTok{x =}\NormalTok{ Year\_of\_Release, }\AttributeTok{y =}\NormalTok{ NA\_total, }\AttributeTok{fill =}\NormalTok{ Platform)) }\SpecialCharTok{+}
    \FunctionTok{geom\_bar}\NormalTok{(}\AttributeTok{stat =} \StringTok{"identity"}\NormalTok{) }\SpecialCharTok{+} \FunctionTok{labs}\NormalTok{(}\AttributeTok{title =} \StringTok{"Total NA sales by Video Game Platform (2010{-}2016)"}\NormalTok{,}
    \AttributeTok{x =} \StringTok{"Year"}\NormalTok{, }\AttributeTok{y =} \StringTok{"NA Sales}\SpecialCharTok{\textbackslash{}n}\StringTok{(In millions)"}\NormalTok{)}
\end{Highlighting}
\end{Shaded}

\includegraphics{Video-Game-Sales-Study--Phase-1-_files/figure-latex/unnamed-chunk-30-1.pdf}

\begin{Shaded}
\begin{Highlighting}[]
\CommentTok{\# labs() is used to changes the labels in various parts of the chart. Both the}
\CommentTok{\# X and Y axis changing their names, as well as the title.}
\end{Highlighting}
\end{Shaded}

This code creates a bar graph that breaks up the total amount of NA
sales for each of the different years of each different Platform.

Analysis: Overall, from these graphs we can see that Xbox is the most
profitable between all these platforms. However, compared to 2010 when
it was at its highest, we are able to see that the sales for Xbox has
been steadily declining same goes for they other platforms as well. But,
compared to other platforms PC has had the least amount of sales, and
have seen that by 2016 PlayStation had they highest sells.

5.2 To determine the most profitable platform globally we must first
analyze the total sales made between 2010 and 2016 as well as analyzing
the rate of change between these platforms.

\begin{Shaded}
\begin{Highlighting}[]
\CommentTok{\# We use the ggplot2 package to create our graph, the aes() function is used to}
\CommentTok{\# set the aesthetics of the plot. we set our x value to Platform, and our y}
\CommentTok{\# value to glbl\_total.}
\FunctionTok{ggplot}\NormalTok{(platform\_glbl\_sales, }\FunctionTok{aes}\NormalTok{(}\AttributeTok{x =}\NormalTok{ Platform, }\AttributeTok{y =}\NormalTok{ glbl\_total, }\AttributeTok{color =}\NormalTok{ Platform)) }\SpecialCharTok{+}
    \FunctionTok{geom\_point}\NormalTok{(}\AttributeTok{stat =} \StringTok{"identity"}\NormalTok{) }\SpecialCharTok{+} \FunctionTok{geom\_segment}\NormalTok{(}\FunctionTok{aes}\NormalTok{(}\AttributeTok{x =}\NormalTok{ Platform, }\AttributeTok{xend =}\NormalTok{ Platform,}
    \AttributeTok{y =} \DecValTok{0}\NormalTok{, }\AttributeTok{yend =}\NormalTok{ glbl\_total)) }\SpecialCharTok{+} \FunctionTok{labs}\NormalTok{(}\AttributeTok{title =} \StringTok{"Global Sales by Platform"}\NormalTok{, }\AttributeTok{x =} \StringTok{"Platform"}\NormalTok{,}
    \AttributeTok{y =} \StringTok{"Global Sales}\SpecialCharTok{\textbackslash{}n}\StringTok{(in millions)"}\NormalTok{)}
\end{Highlighting}
\end{Shaded}

\includegraphics{Video-Game-Sales-Study--Phase-1-_files/figure-latex/unnamed-chunk-31-1.pdf}

\begin{Shaded}
\begin{Highlighting}[]
\CommentTok{\# geom\_point() sets a point at the value of each platforms global sales.}
\CommentTok{\# geom\_segment() creates lines at our platforms and reaches its total amount.}
\CommentTok{\# We use labs() to rename the x and y axis as well as our title.}
\end{Highlighting}
\end{Shaded}

This code generates a lollipop graph in which it displays the total
global sales of each platform in 2010 to 2016.

\begin{Shaded}
\begin{Highlighting}[]
\FunctionTok{ggplot}\NormalTok{(}\AttributeTok{data =}\NormalTok{ platform\_SalesbyYear, }\FunctionTok{aes}\NormalTok{(}\AttributeTok{x =} \DecValTok{0}\NormalTok{, }\AttributeTok{y =}\NormalTok{ glbl\_total, }\AttributeTok{fill =}\NormalTok{ Platform)) }\SpecialCharTok{+}
    \FunctionTok{geom\_col}\NormalTok{(}\AttributeTok{position =} \StringTok{"fill"}\NormalTok{) }\SpecialCharTok{+} \FunctionTok{facet\_wrap}\NormalTok{(}\SpecialCharTok{\textasciitilde{}}\NormalTok{Year\_of\_Release) }\SpecialCharTok{+} \FunctionTok{coord\_polar}\NormalTok{(}\AttributeTok{theta =} \StringTok{"y"}\NormalTok{) }\SpecialCharTok{+}
    \FunctionTok{theme\_void}\NormalTok{() }\SpecialCharTok{+} \FunctionTok{labs}\NormalTok{(}\AttributeTok{title =} \StringTok{"Global Sales by Platform"}\NormalTok{)}
\end{Highlighting}
\end{Shaded}

\includegraphics{Video-Game-Sales-Study--Phase-1-_files/figure-latex/unnamed-chunk-32-1.pdf}
This code generates pie charts which shows the difference in sales
between each platform from 2010 to 2016.

\begin{Shaded}
\begin{Highlighting}[]
\CommentTok{\# ggplot() is used to create the graph, and set the aesthetics of the graph.}
\CommentTok{\# geom\_point() is used to create a scatter plot of each platforms sales from}
\CommentTok{\# 2010 to 2016.}
\FunctionTok{ggplot}\NormalTok{(}\AttributeTok{data =}\NormalTok{ platform\_SalesbyYear, }\FunctionTok{aes}\NormalTok{(}\AttributeTok{x =}\NormalTok{ Year\_of\_Release, }\AttributeTok{y =}\NormalTok{ glbl\_total, }\AttributeTok{color =}\NormalTok{ Platform)) }\SpecialCharTok{+}
    \FunctionTok{geom\_point}\NormalTok{() }\SpecialCharTok{+} \FunctionTok{geom\_line}\NormalTok{(}\FunctionTok{aes}\NormalTok{(}\AttributeTok{group =}\NormalTok{ Platform)) }\SpecialCharTok{+} \FunctionTok{labs}\NormalTok{(}\AttributeTok{title =} \StringTok{"Global Video Game Sales (2010{-}2016)"}\NormalTok{,}
    \AttributeTok{x =} \StringTok{"Year"}\NormalTok{, }\AttributeTok{y =} \StringTok{"Global Sales}\SpecialCharTok{\textbackslash{}n}\StringTok{(in millions)"}\NormalTok{)}
\end{Highlighting}
\end{Shaded}

\includegraphics{Video-Game-Sales-Study--Phase-1-_files/figure-latex/unnamed-chunk-33-1.pdf}

\begin{Shaded}
\begin{Highlighting}[]
\CommentTok{\# Geom\_line is used to create a line graph and we set the aesthetic of the}
\CommentTok{\# graph to group by Platform. labs() is used to change the names of our x and y}
\CommentTok{\# axis as well as the title.}
\end{Highlighting}
\end{Shaded}

This code generates a line graph that shows the rate of
increase/decrease of each Platform from 2010 to 2016.

Analysis: From our charts we can see that globally PlayStation has
dominated with the most sales, and since 2010 the sales of PlayStation
has had far more sales compared to other Platforms. We also see that the
sales of PC has been low compared to other platforms and that the sales
of Nintendo and Xbox has been steadily decreasing, with Nintendo
decreasing the most in sales in the past 6 years.

5.3 We now wan to create some visualizations for our third question to
better understand the NA sales for each genre in 2010 to 2016.

\begin{Shaded}
\begin{Highlighting}[]
\CommentTok{\# We use the ggplot2 package to create a Bar graph that easily represents the}
\CommentTok{\# highest selling video game Genre for North American sales. We use the aes()}
\CommentTok{\# function to label the axis’, as well as using fill for discernment between}
\CommentTok{\# Genres, we use the reorder() function to arrange the attributes from least to}
\CommentTok{\# greatest. geom\_bar(stat=”identity”) makes a bar chart and makes the height}
\CommentTok{\# proportional to the number of cases in each group. Stat identity displays the}
\CommentTok{\# sum of values in the Sales(NA) column, grouped by Genre.}
\FunctionTok{ggplot}\NormalTok{(}\AttributeTok{data =}\NormalTok{ genre\_na\_sales, }\FunctionTok{aes}\NormalTok{(}\AttributeTok{x =} \FunctionTok{reorder}\NormalTok{(Genre, genre\_na\_total), }\AttributeTok{y =}\NormalTok{ genre\_na\_total,}
    \AttributeTok{fill =}\NormalTok{ Genre)) }\SpecialCharTok{+} \FunctionTok{geom\_bar}\NormalTok{(}\AttributeTok{stat =} \StringTok{"identity"}\NormalTok{) }\SpecialCharTok{+} \FunctionTok{labs}\NormalTok{(}\AttributeTok{title =} \StringTok{"NA sales by Genre"}\NormalTok{,}
    \AttributeTok{x =} \StringTok{"Genre"}\NormalTok{, }\AttributeTok{y =} \StringTok{"NA Sales"}\NormalTok{) }\SpecialCharTok{+} \FunctionTok{theme}\NormalTok{(}\AttributeTok{axis.text.x =} \FunctionTok{element\_text}\NormalTok{(}\AttributeTok{angle =} \DecValTok{45}\NormalTok{, }\AttributeTok{hjust =} \DecValTok{1}\NormalTok{))}
\end{Highlighting}
\end{Shaded}

\includegraphics{Video-Game-Sales-Study--Phase-1-_files/figure-latex/unnamed-chunk-34-1.pdf}

\begin{Shaded}
\begin{Highlighting}[]
\CommentTok{\# labs() is used to changes the labels in various parts of the chart. Both the}
\CommentTok{\# X and Y axis changing their names, as well as the title. theme() helps change}
\CommentTok{\# non{-}data elements of a chart. In this case adding a change to Genre’s axis by}
\CommentTok{\# utilizing element\_text() to change the look of texts, adding an angle and}
\CommentTok{\# using hjust to control horizontal justification)}
\end{Highlighting}
\end{Shaded}

This code creates a bar graph that displays the overall sum of sales of
each genre between 2010 to 2016.

\begin{Shaded}
\begin{Highlighting}[]
\CommentTok{\# We convert our numeric values in genre\_na\_sales into a percentage of the}
\CommentTok{\# entire sum and store it into pie\_labels.}
\NormalTok{pie\_labels }\OtherTok{\textless{}{-}} \FunctionTok{paste0}\NormalTok{(}\FunctionTok{round}\NormalTok{(}\DecValTok{100} \SpecialCharTok{*}\NormalTok{ genre\_na\_sales}\SpecialCharTok{$}\NormalTok{genre\_na\_total}\SpecialCharTok{/}\FunctionTok{sum}\NormalTok{(genre\_na\_sales}\SpecialCharTok{$}\NormalTok{genre\_na\_total),}
    \DecValTok{2}\NormalTok{), }\StringTok{"\%"}\NormalTok{)}
\CommentTok{\# We use ggplot() to create the graph, the aes() function set’s the aesthetic.}
\CommentTok{\# Used nothing for X, but for Y we used NA Sales so that it would fill}
\CommentTok{\# proportionately.We use the reorder() function to reorder the different}
\CommentTok{\# Genre\textquotesingle{}s from least to greatest. We then filled it by Genre using a rainbow}
\CommentTok{\# palette. the geom\_col() function is used to create black lines in between our}
\CommentTok{\# “Slices”.}
\FunctionTok{ggplot}\NormalTok{(genre\_na\_sales, }\FunctionTok{aes}\NormalTok{(}\AttributeTok{x =} \StringTok{""}\NormalTok{, }\AttributeTok{y =}\NormalTok{ genre\_na\_total, }\AttributeTok{fill =}\NormalTok{ Genre)) }\SpecialCharTok{+} \FunctionTok{geom\_col}\NormalTok{(}\AttributeTok{color =} \StringTok{"black"}\NormalTok{) }\SpecialCharTok{+}
    \FunctionTok{geom\_text}\NormalTok{(}\FunctionTok{aes}\NormalTok{(}\AttributeTok{label =}\NormalTok{ pie\_labels), }\AttributeTok{position =} \FunctionTok{position\_stack}\NormalTok{(}\AttributeTok{vjust =} \FloatTok{0.5}\NormalTok{)) }\SpecialCharTok{+}
    \FunctionTok{coord\_polar}\NormalTok{(}\AttributeTok{theta =} \StringTok{"y"}\NormalTok{) }\SpecialCharTok{+} \FunctionTok{labs}\NormalTok{(}\AttributeTok{y =} \StringTok{"NA Sales"}\NormalTok{, }\AttributeTok{title =} \StringTok{"NA Sales by Genre"}\NormalTok{)}
\end{Highlighting}
\end{Shaded}

\includegraphics{Video-Game-Sales-Study--Phase-1-_files/figure-latex/unnamed-chunk-35-1.pdf}

\begin{Shaded}
\begin{Highlighting}[]
\CommentTok{\# geom\_text is used along with aes() embedded to set labels to pie labels and}
\CommentTok{\# position them accordingly. coord\_polar is used because a pie chart is}
\CommentTok{\# actually just a stacked bar chart in polar coordinates. Then, we use the labs}
\CommentTok{\# function to change the name of the y axis and title on the graph.}
\end{Highlighting}
\end{Shaded}

This code creates a pie graph, then displays the difference in NA sales
each genre made out of 100\% of the total sales made between 2010 to
2016

\begin{Shaded}
\begin{Highlighting}[]
\CommentTok{\# ggplot() is used to create the graph, and set the aesthetics of the graph.}
\CommentTok{\# geom\_point() is used to create a scatter plot of each platforms sales from}
\CommentTok{\# 2010 to 2016.}
\FunctionTok{ggplot}\NormalTok{(}\AttributeTok{data =}\NormalTok{ genre\_SalesbyYear, }\FunctionTok{aes}\NormalTok{(}\AttributeTok{x =}\NormalTok{ Year\_of\_Release, }\AttributeTok{y =}\NormalTok{ NA\_total, }\AttributeTok{color =}\NormalTok{ Genre)) }\SpecialCharTok{+}
    \FunctionTok{geom\_point}\NormalTok{() }\SpecialCharTok{+} \FunctionTok{geom\_line}\NormalTok{(}\FunctionTok{aes}\NormalTok{(}\AttributeTok{group =}\NormalTok{ Genre)) }\SpecialCharTok{+} \FunctionTok{labs}\NormalTok{(}\AttributeTok{title =} \StringTok{"NA Video Game Sales (2010{-}2016)"}\NormalTok{,}
    \AttributeTok{x =} \StringTok{"Year"}\NormalTok{, }\AttributeTok{y =} \StringTok{"NA Sales}\SpecialCharTok{\textbackslash{}n}\StringTok{(in millions)"}\NormalTok{)}
\end{Highlighting}
\end{Shaded}

\includegraphics{Video-Game-Sales-Study--Phase-1-_files/figure-latex/unnamed-chunk-36-1.pdf}

\begin{Shaded}
\begin{Highlighting}[]
\CommentTok{\# Geom\_line is used to create a line graph and we set the aesthetic of the}
\CommentTok{\# graph to group by Genre. labs() is used to change the names of our x and y}
\CommentTok{\# axis as well as the title.}
\end{Highlighting}
\end{Shaded}

This code creates a line graph that displays the change of NA sales
based on each genre over the past 6 years.

Analysis: Overall, we assume that Action is the most profitable video
game genre compared to the other games, however the sales of action has
dramatically decreased since 2013. While Shooter games has also
decreased it was the most profitable video game genre by 2016.

5.4 We now see the total global revenue based on Genre to get a visual
for which Genre was more profitable all throughout 2010-2016.

\begin{Shaded}
\begin{Highlighting}[]
\CommentTok{\# We use the ggplot2 package to create a Bar graph that easily represents the}
\CommentTok{\# highest globally selling video game genre. We use the aes() function to label}
\CommentTok{\# the axis’, as well as using fill for discernment between Genres,}
\CommentTok{\# geom\_bar(stat=”identity”) makes a bar chart and makes the height proportional}
\CommentTok{\# to the number of cases in each group. Stat identity displays the sum of}
\CommentTok{\# values in the Sales(NA) column, grouped by Genre}
\FunctionTok{ggplot}\NormalTok{(}\AttributeTok{data =}\NormalTok{ genre\_glbl\_sales, }\FunctionTok{aes}\NormalTok{(}\AttributeTok{x =}\NormalTok{ Genre, }\AttributeTok{y =}\NormalTok{ genre\_glbl\_total, }\AttributeTok{fill =}\NormalTok{ Genre)) }\SpecialCharTok{+}
    \FunctionTok{geom\_bar}\NormalTok{(}\AttributeTok{stat =} \StringTok{"identity"}\NormalTok{) }\SpecialCharTok{+} \FunctionTok{labs}\NormalTok{(}\AttributeTok{title =} \StringTok{"Global Sales by Genre"}\NormalTok{, }\AttributeTok{x =} \StringTok{"Genre"}\NormalTok{,}
    \AttributeTok{y =} \StringTok{"Global Sales}\SpecialCharTok{\textbackslash{}n}\StringTok{(In millions)"}\NormalTok{) }\SpecialCharTok{+} \FunctionTok{theme}\NormalTok{(}\AttributeTok{axis.text.x =} \FunctionTok{element\_text}\NormalTok{(}\AttributeTok{angle =} \DecValTok{45}\NormalTok{,}
    \AttributeTok{hjust =} \DecValTok{1}\NormalTok{))}
\end{Highlighting}
\end{Shaded}

\includegraphics{Video-Game-Sales-Study--Phase-1-_files/figure-latex/unnamed-chunk-37-1.pdf}

\begin{Shaded}
\begin{Highlighting}[]
\CommentTok{\# labs() is used to changes the labels in various parts of the chart. Both the}
\CommentTok{\# X and Y axis changing their names, as well as the title. theme() helps change}
\CommentTok{\# non{-}data elements of a chart. In this case adding a change to Genre’s axis by}
\CommentTok{\# utilizing element\_text() to change the look of texts, adding an angle and}
\CommentTok{\# using hjust to control horizontal justification}
\end{Highlighting}
\end{Shaded}

This code generates a bar graph that displays the total Global sales of
each video game genre between 2010 to 2016.

\begin{Shaded}
\begin{Highlighting}[]
\CommentTok{\# We convert our numerical values into percentages, then store into pvgsg.}
\NormalTok{PVGSG }\OtherTok{\textless{}{-}}\NormalTok{ genre\_glbl\_sales }\SpecialCharTok{\%\textgreater{}\%}
    \FunctionTok{mutate}\NormalTok{(}\AttributeTok{percentage =} \FunctionTok{paste0}\NormalTok{(}\FunctionTok{round}\NormalTok{(genre\_glbl\_total}\SpecialCharTok{/}\FunctionTok{sum}\NormalTok{(genre\_glbl\_total) }\SpecialCharTok{*} \DecValTok{100}\NormalTok{,}
        \DecValTok{2}\NormalTok{), }\StringTok{"\%"}\NormalTok{))}
\NormalTok{PVGSG }\OtherTok{\textless{}{-}}\NormalTok{ PVGSG }\SpecialCharTok{\%\textgreater{}\%}
    \FunctionTok{arrange}\NormalTok{(}\FunctionTok{desc}\NormalTok{(genre\_glbl\_total))}

\CommentTok{\# We use ggplot() to create a bar chart, then use aes() to set the parameters}
\CommentTok{\# for x and y axis, and use geom\_bar() to create a bar chart and then set the}
\CommentTok{\# style for to fill.}
\FunctionTok{ggplot}\NormalTok{(PVGSG, }\FunctionTok{aes}\NormalTok{(}\AttributeTok{fill =}\NormalTok{ Genre, }\AttributeTok{y =}\NormalTok{ genre\_glbl\_total, }\AttributeTok{x =} \StringTok{""}\NormalTok{)) }\SpecialCharTok{+} \FunctionTok{geom\_bar}\NormalTok{(}\AttributeTok{position =} \StringTok{"fill"}\NormalTok{,}
    \AttributeTok{stat =} \StringTok{"identity"}\NormalTok{)}
\end{Highlighting}
\end{Shaded}

\includegraphics{Video-Game-Sales-Study--Phase-1-_files/figure-latex/unnamed-chunk-38-1.pdf}
This code displays a stacked bar graph with a couple of functions at
play.

\begin{Shaded}
\begin{Highlighting}[]
\CommentTok{\# We convert our numeric values in genre\_glbl\_sales into a percentage of the}
\CommentTok{\# entire sum and store it into pie\_labels2.}
\NormalTok{pie\_labels2 }\OtherTok{\textless{}{-}} \FunctionTok{paste0}\NormalTok{(}\FunctionTok{round}\NormalTok{(}\DecValTok{100} \SpecialCharTok{*}\NormalTok{ genre\_glbl\_sales}\SpecialCharTok{$}\NormalTok{genre\_glbl\_total}\SpecialCharTok{/}\FunctionTok{sum}\NormalTok{(genre\_glbl\_sales}\SpecialCharTok{$}\NormalTok{genre\_glbl\_total),}
    \DecValTok{2}\NormalTok{), }\StringTok{"\%"}\NormalTok{)}
\CommentTok{\# We use ggplot() to create the graph, the aes() function set’s the aesthetic.}
\CommentTok{\# Used nothing for X, but for Y we used genre\_glbl\_total so that it would fill}
\CommentTok{\# proportionately.We use the reorder() function to reorder the different}
\CommentTok{\# Genre\textquotesingle{}s from least to greatest. We then filled it by Genre using a rainbow}
\CommentTok{\# palette. the geom\_col() function is used to create black lines in between our}
\CommentTok{\# “Slices”.}
\FunctionTok{ggplot}\NormalTok{(genre\_glbl\_sales, }\FunctionTok{aes}\NormalTok{(}\AttributeTok{x =} \StringTok{""}\NormalTok{, }\AttributeTok{y =}\NormalTok{ genre\_glbl\_total, }\AttributeTok{fill =}\NormalTok{ Genre)) }\SpecialCharTok{+} \FunctionTok{geom\_col}\NormalTok{(}\AttributeTok{color =} \StringTok{"black"}\NormalTok{) }\SpecialCharTok{+}
    \FunctionTok{geom\_text}\NormalTok{(}\FunctionTok{aes}\NormalTok{(}\AttributeTok{label =}\NormalTok{ pie\_labels2), }\AttributeTok{position =} \FunctionTok{position\_stack}\NormalTok{(}\AttributeTok{vjust =} \FloatTok{0.5}\NormalTok{)) }\SpecialCharTok{+}
    \FunctionTok{coord\_polar}\NormalTok{(}\AttributeTok{theta =} \StringTok{"y"}\NormalTok{)}
\end{Highlighting}
\end{Shaded}

\includegraphics{Video-Game-Sales-Study--Phase-1-_files/figure-latex/unnamed-chunk-39-1.pdf}

\begin{Shaded}
\begin{Highlighting}[]
\CommentTok{\# geom\_text is used along with aes() embedded to set labels to pie labels and}
\CommentTok{\# position them accordingly. coord\_polar is used because a pie chart is}
\CommentTok{\# actually just a stacked bar chart in polar coordinates. Then, we use the labs}
\CommentTok{\# function to change the name of the y axis and title on the graph.}
\end{Highlighting}
\end{Shaded}

This code creates a pie chart in which it displays the sum of global
sale of video games between 2010 to 2016 in a percentage of the total
sales made by genre.

\begin{Shaded}
\begin{Highlighting}[]
\CommentTok{\# ggplot() is used to create the graph, and set the aesthetics of the graph.}
\CommentTok{\# geom\_point() is used to create a scatter plot of each platforms sales from}
\CommentTok{\# 2010 to 2016.}
\FunctionTok{ggplot}\NormalTok{(}\AttributeTok{data =}\NormalTok{ genre\_SalesbyYear, }\FunctionTok{aes}\NormalTok{(}\AttributeTok{x =}\NormalTok{ Year\_of\_Release, }\AttributeTok{y =}\NormalTok{ glbl\_total, }\AttributeTok{color =}\NormalTok{ Genre)) }\SpecialCharTok{+}
    \FunctionTok{geom\_point}\NormalTok{() }\SpecialCharTok{+} \FunctionTok{geom\_line}\NormalTok{(}\FunctionTok{aes}\NormalTok{(}\AttributeTok{group =}\NormalTok{ Genre)) }\SpecialCharTok{+} \FunctionTok{labs}\NormalTok{(}\AttributeTok{title =} \StringTok{"Global Video Game Sales (2010{-}2016)"}\NormalTok{,}
    \AttributeTok{x =} \StringTok{"Year"}\NormalTok{, }\AttributeTok{y =} \StringTok{"Global Sales}\SpecialCharTok{\textbackslash{}n}\StringTok{(in millions)"}\NormalTok{)}
\end{Highlighting}
\end{Shaded}

\includegraphics{Video-Game-Sales-Study--Phase-1-_files/figure-latex/unnamed-chunk-40-1.pdf}

\begin{Shaded}
\begin{Highlighting}[]
\CommentTok{\# Geom\_line is used to create a line graph and we set the aesthetic of the}
\CommentTok{\# graph to group by Genre. labs() is used to change the names of our x and y}
\CommentTok{\# axis as well as the title.}
\end{Highlighting}
\end{Shaded}

This line creates a line chart that shows the decrease/increase of
gloabl video game sales each year.

Analysis: Just like our previous analysis with question 3, we see that
similar to NA sales Action is also the most profitable genre, however
over the course of 6 years the sales of Action games have decreased
dramatically and eventually shooter has the most sales in 2016.

5.5 We know the global and NA revenue for each of the top five
developers, but now we want to create some visuals, to get a better
understanding of their profit within 2010 to 2016.

\begin{Shaded}
\begin{Highlighting}[]
\CommentTok{\# We use the ggplot2 package to create our graph, the aes() function is used to}
\CommentTok{\# set the aesthetics of the plot. we set our x value to Developer and our Y}
\CommentTok{\# value to NA sale. we also use the reorder() function to reorder the graph}
\CommentTok{\# from smallest to greatest.}

\FunctionTok{ggplot}\NormalTok{(dev\_global, }\FunctionTok{aes}\NormalTok{(}\AttributeTok{x =} \FunctionTok{reorder}\NormalTok{(Developer, global\_total), }\AttributeTok{y =}\NormalTok{ global\_total, }\AttributeTok{fill =}\NormalTok{ Developer)) }\SpecialCharTok{+}
    \FunctionTok{geom\_bar}\NormalTok{(}\AttributeTok{stat =} \StringTok{"identity"}\NormalTok{) }\SpecialCharTok{+} \FunctionTok{theme}\NormalTok{(}\AttributeTok{axis.text.x =} \FunctionTok{element\_text}\NormalTok{(}\AttributeTok{angle =} \DecValTok{45}\NormalTok{, }\AttributeTok{hjust =} \DecValTok{1}\NormalTok{)) }\SpecialCharTok{+}
    \FunctionTok{labs}\NormalTok{(}\AttributeTok{title =} \StringTok{"Top 5 Developer Global Sales"}\NormalTok{, }\AttributeTok{x =} \StringTok{"Developer"}\NormalTok{, }\AttributeTok{y =} \StringTok{"Global Sales}\SpecialCharTok{\textbackslash{}n}\StringTok{(In millions)"}\NormalTok{)}
\end{Highlighting}
\end{Shaded}

\includegraphics{Video-Game-Sales-Study--Phase-1-_files/figure-latex/unnamed-chunk-41-1.pdf}

\begin{Shaded}
\begin{Highlighting}[]
\CommentTok{\# The geom\_bar() function creates a bar plot using the \textquotesingle{}identity\textquotesingle{} statistical}
\CommentTok{\# transformation.The theme() function is used to customize the appearance of}
\CommentTok{\# the plot. The axis.text.x argument is set to element\_text() to change the}
\CommentTok{\# angle of the x{-}axis labels to 45 degrees and set the horizontal justification}
\CommentTok{\# to 1 (right align).The labs() function sets the plot title to \textquotesingle{}Top 5}
\CommentTok{\# Developers by Global Sales\textquotesingle{}, and labels the x{-}axis \textquotesingle{}Developer\textquotesingle{} and the y{-}axis}
\CommentTok{\# \textquotesingle{}Global Game Sales (in millions)\textquotesingle{}.}
\end{Highlighting}
\end{Shaded}

Overall, this code generates a bar chart that ranks the top 5 video game
developers by their total global sales across all regions. The bars are
ordered by the sum of sales from all regions for each developer. The
x-axis labels are angled and right aligned for better readability.

\begin{Shaded}
\begin{Highlighting}[]
\CommentTok{\# we convert our numeric values in genre\_glbl\_sales into a percentage of the}
\CommentTok{\# entire sum and store it into pie\_labels2.}
\NormalTok{pie\_labels3 }\OtherTok{\textless{}{-}} \FunctionTok{paste0}\NormalTok{(}\FunctionTok{round}\NormalTok{(}\DecValTok{100} \SpecialCharTok{*}\NormalTok{ dev\_global}\SpecialCharTok{$}\NormalTok{global\_total}\SpecialCharTok{/}\FunctionTok{sum}\NormalTok{(dev\_global}\SpecialCharTok{$}\NormalTok{global\_total),}
    \DecValTok{2}\NormalTok{), }\StringTok{"\%"}\NormalTok{)}
\CommentTok{\# The aes() function is setting the aesthetics of the plot. The x{-}axis is set}
\CommentTok{\# to an empty string, which means there will be no x{-}axis label. The y{-}axis is}
\CommentTok{\# set to the sum of sales from all regions. The fill aesthetic is set to the}
\CommentTok{\# Developer column. The geom\_col() function creates a pie chart using the}
\CommentTok{\# \textquotesingle{}identity\textquotesingle{} statistical transformation. The width argument is set to 1 to}
\CommentTok{\# remove the space between the bars, and the color argument is set to \textquotesingle{}black\textquotesingle{}}
\CommentTok{\# to add a white border around each bar. The coord\_polar() function is used to}
\CommentTok{\# convert the plot to a polar coordinate system. The \textquotesingle{}y\textquotesingle{} argument specifies}
\CommentTok{\# that the y{-}axis values should be used to determine the radial distance of}
\CommentTok{\# each bar from the center of the plot. The start argument is set to 0 to align}
\CommentTok{\# the first bar with the 12 o\textquotesingle{}clock position.}
\FunctionTok{ggplot}\NormalTok{(dev\_global, }\FunctionTok{aes}\NormalTok{(}\AttributeTok{x =} \StringTok{""}\NormalTok{, }\AttributeTok{y =}\NormalTok{ global\_total, }\AttributeTok{fill =}\NormalTok{ Developer)) }\SpecialCharTok{+} \FunctionTok{geom\_col}\NormalTok{(}\AttributeTok{stat =} \StringTok{"identity"}\NormalTok{,}
    \AttributeTok{width =} \DecValTok{1}\NormalTok{, }\AttributeTok{color =} \StringTok{"black"}\NormalTok{) }\SpecialCharTok{+} \FunctionTok{coord\_polar}\NormalTok{(}\AttributeTok{theta =} \StringTok{"y"}\NormalTok{) }\SpecialCharTok{+} \FunctionTok{theme\_void}\NormalTok{() }\SpecialCharTok{+} \FunctionTok{geom\_text}\NormalTok{(}\FunctionTok{aes}\NormalTok{(}\AttributeTok{label =}\NormalTok{ pie\_labels3),}
    \AttributeTok{position =} \FunctionTok{position\_stack}\NormalTok{(}\AttributeTok{vjust =} \FloatTok{0.5}\NormalTok{)) }\SpecialCharTok{+} \FunctionTok{labs}\NormalTok{(}\AttributeTok{title =} \StringTok{" Top 5 Developer Sales"}\NormalTok{)}
\end{Highlighting}
\end{Shaded}

\begin{verbatim}
## Warning in geom_col(stat = "identity", width = 1, color = "black"): Ignoring
## unknown parameters: `stat`
\end{verbatim}

\includegraphics{Video-Game-Sales-Study--Phase-1-_files/figure-latex/unnamed-chunk-42-1.pdf}

\begin{Shaded}
\begin{Highlighting}[]
\CommentTok{\# The geom\_text() function is used to sum up the percentage each of these five}
\CommentTok{\# Developers make globally. The theme\_void() function removes all the axis}
\CommentTok{\# labels, ticks, and grid lines, leaving only the bars.The labs() function sets}
\CommentTok{\# the plot title to \textquotesingle{}Top 5 Developer Sales\textquotesingle{}.}
\end{Highlighting}
\end{Shaded}

Overall, this code generates a pie chart that ranks the top 5 video game
developers by their total global sales across all regions. The bars are
arranged radially, with the outermost bar representing the developer
with the highest total sales. The plot has no axis labels or grid lines,
giving it a minimalist look.

Analysis: The first bar graph shows the order of the top 5 game
developers in terms of the number of game copies sold. In ascending
order, the top 5 game developers are Ubisoft, EA Sports, Nintendo,
Treyarch, and Rockstar North. The pie chart provides another
visualization for this ranking.

5.6 We are now going to create some visuals for our sixth question in
order to analyze our findings. developer\_genre\_sales

\begin{Shaded}
\begin{Highlighting}[]
\CommentTok{\# We use the ggplot2 package in R to show the top 5 video game genres for each}
\CommentTok{\# of the top 5 video game developers in terms of global sales. The aes()}
\CommentTok{\# function is used to set the aesthetics of the plot. The x{-}axis is set to the}
\CommentTok{\# Developer column, the y{-}axis is set to the Global\_Total column, and the fill}
\CommentTok{\# aesthetic is set to the Genre column. The geom\_bar() function creates a bar}
\CommentTok{\# plot using the \textquotesingle{}identity\textquotesingle{} statistical transformation, which means the bar}
\CommentTok{\# heights correspond to the values in the Global\_Total column.}
\FunctionTok{ggplot}\NormalTok{(df, }\FunctionTok{aes}\NormalTok{(}\AttributeTok{x =}\NormalTok{ Developer, }\AttributeTok{y =}\NormalTok{ developer\_glbl\_total, }\AttributeTok{fill =}\NormalTok{ Genre)) }\SpecialCharTok{+} \FunctionTok{geom\_bar}\NormalTok{(}\AttributeTok{stat =} \StringTok{"identity"}\NormalTok{) }\SpecialCharTok{+}
    \FunctionTok{theme}\NormalTok{(}\AttributeTok{axis.text.x =} \FunctionTok{element\_text}\NormalTok{(}\AttributeTok{angle =} \DecValTok{45}\NormalTok{, }\AttributeTok{hjust =} \DecValTok{1}\NormalTok{)) }\SpecialCharTok{+} \FunctionTok{labs}\NormalTok{(}\AttributeTok{title =} \StringTok{"Top 5 Developer Sales by Genre"}\NormalTok{,}
    \AttributeTok{x =} \StringTok{"Developer"}\NormalTok{, }\AttributeTok{y =} \StringTok{"Global Game Sales}\SpecialCharTok{\textbackslash{}n}\StringTok{(In millions)"}\NormalTok{)}
\end{Highlighting}
\end{Shaded}

\includegraphics{Video-Game-Sales-Study--Phase-1-_files/figure-latex/unnamed-chunk-43-1.pdf}

\begin{Shaded}
\begin{Highlighting}[]
\CommentTok{\# The theme() function sets the x{-}axis text to a 45{-}degree angle for better}
\CommentTok{\# readability, and the labs() function sets the plot title, x{-}axis label, and}
\CommentTok{\# y{-}axis label.}
\end{Highlighting}
\end{Shaded}

Overall, this code generates a stacked bar chart that shows the
contribution of each video game genre to the global sales of the top 5
video game developers. Each bar is divided into segments corresponding
to the sales of each genre, and the height of each bar corresponds to
the total global sales of the developer.

\begin{Shaded}
\begin{Highlighting}[]
\CommentTok{\# The aes() function is used to set the aesthetics of the plot. The x{-}axis is}
\CommentTok{\# set to 0, the y{-}axis is set to the Global\_Total column, and the fill}
\CommentTok{\# aesthetic is set to the Genre column. The geom\_col() function creates a}
\CommentTok{\# column plot with position set to \textquotesingle{}fill\textquotesingle{}, which stacks the bars so that each}
\CommentTok{\# bar fills the available vertical space.}
\FunctionTok{ggplot}\NormalTok{(}\AttributeTok{data =}\NormalTok{ df, }\FunctionTok{aes}\NormalTok{(}\AttributeTok{x =} \DecValTok{0}\NormalTok{, }\AttributeTok{y =}\NormalTok{ developer\_glbl\_total, }\AttributeTok{fill =}\NormalTok{ Genre)) }\SpecialCharTok{+} \FunctionTok{geom\_col}\NormalTok{(}\AttributeTok{position =} \StringTok{"fill"}\NormalTok{) }\SpecialCharTok{+}
    \FunctionTok{facet\_wrap}\NormalTok{(}\SpecialCharTok{\textasciitilde{}}\NormalTok{Developer) }\SpecialCharTok{+} \FunctionTok{coord\_polar}\NormalTok{(}\AttributeTok{theta =} \StringTok{"y"}\NormalTok{) }\SpecialCharTok{+} \FunctionTok{theme\_void}\NormalTok{() }\SpecialCharTok{+} \FunctionTok{labs}\NormalTok{(}\AttributeTok{title =} \StringTok{"Top 5 Developer sales by Genre"}\NormalTok{)}
\end{Highlighting}
\end{Shaded}

\includegraphics{Video-Game-Sales-Study--Phase-1-_files/figure-latex/unnamed-chunk-44-1.pdf}

\begin{Shaded}
\begin{Highlighting}[]
\CommentTok{\# The facet\_wrap() function is used to split the plot into multiple panels, one}
\CommentTok{\# for each developer. The coord\_polar() function converts the plot to polar}
\CommentTok{\# coordinates, which makes it easier to compare the relative contributions of}
\CommentTok{\# each video game genre across the different developers. The theme\_void()}
\CommentTok{\# function sets the plot to have no background or axis labels, and the labs()}
\CommentTok{\# function sets the plot title.}
\end{Highlighting}
\end{Shaded}

Overall, this code generates a polar bar chart that shows the relative
contribution of each video game genre to the global sales of the top 5
video game developers. The chart is split into multiple panels, one for
each developer, making it easy to compare the contribution of each genre
across the different developers.

\begin{Shaded}
\begin{Highlighting}[]
\CommentTok{\# The aes() function is used to set the aesthetics of the plot. The x{-}axis is}
\CommentTok{\# set to Genre, the y{-}axis is set to the Global\_Total column, and the color}
\CommentTok{\# aesthetic is set to the Developer column. The geom\_point() function creates a}
\CommentTok{\# scatter plot graph.}
\FunctionTok{ggplot}\NormalTok{(}\AttributeTok{data =}\NormalTok{ df, }\FunctionTok{aes}\NormalTok{(}\AttributeTok{x =}\NormalTok{ Genre, }\AttributeTok{y =}\NormalTok{ developer\_glbl\_total, }\AttributeTok{colour =}\NormalTok{ Developer)) }\SpecialCharTok{+}
    \FunctionTok{geom\_point}\NormalTok{() }\SpecialCharTok{+} \FunctionTok{labs}\NormalTok{(}\AttributeTok{title =} \StringTok{"Top 5 Developer sales by Genre"}\NormalTok{, }\AttributeTok{x =} \StringTok{"Genre"}\NormalTok{, }\AttributeTok{y =} \StringTok{"Global Game Sales}\SpecialCharTok{\textbackslash{}n}\StringTok{(In millions)"}\NormalTok{) }\SpecialCharTok{+}
    \FunctionTok{coord\_flip}\NormalTok{()}
\end{Highlighting}
\end{Shaded}

\includegraphics{Video-Game-Sales-Study--Phase-1-_files/figure-latex/unnamed-chunk-45-1.pdf}

\begin{Shaded}
\begin{Highlighting}[]
\CommentTok{\# We use the labs() function to label our title, x{-}axis, and y{-}axis. The}
\CommentTok{\# coordflip() flips the x and y axis, so that our x{-}axis value appear on the}
\CommentTok{\# y{-}axis and our y{-}axis values appear on the x{-}axis.}
\end{Highlighting}
\end{Shaded}

Overall this code creates a scatter plot that shows how most categories
are similar in sales. However it also shows that When it comes to Sports
games, EA Games make far more profit then any other categories.

Analysis: These visualizations represent the sales the top 5 game
developers have earned per video game genre. Rockstar North and Treyarch
develop exclusively Action and Shooter games respectively, meaning they
have only sold games in those genres. EA Sports is almost the same
however, they have developed at least one game in the racing genre.
Nonetheless, most of the game copies EA Sports has sold are in the
sports genre. Ubisoft has more diversity than the previous 3 developers,
but it is visible that they have sold mostly action games or games that
do not fall in any of the named genres. Nintendo is visibly the most
diverse out of the 5 developers in terms of genre, with their most
successful genre being platforms.

\end{document}
